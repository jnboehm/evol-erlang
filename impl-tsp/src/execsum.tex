\documentclass[compress]{beamer}

% basically only used for the \to in \item
\usepackage{listings}
\usepackage{qtree}
\usepackage[math]{kurier}
\usepackage{fontspec}
\usepackage[noflama]{hsrm}

\lstset{%
    basicstyle=\small\ttfamily, %
    commentstyle=\color{gray}, %
    stringstyle=\color{Green}, %
    keywordstyle=\bfseries\color{hsrmRed}, %
    frame=tb, % line at Top and one at Bottom
    showstringspaces=false, %
    language=Java, %
    moredelim=[is][\textcolor{gray}]{\%\%}{\%\%}, %colorize opt params
    moredelim=[is][\textcolor{hsrmSec2Dark}]{==}{==}, %To overcome the weird
                                                      %behavior of Strings in
                                                      %beamer
}

\newcommand{\hsrm}{Hochschule {\Medium RheinMain}}

\AtBeginDocument{
\title{Executive Summary 1}
\subtitle{Titanic: Machine Learning from Disaster}
\author{Niklas Böhm \& Marvin Duchmann}
\institute{%
\begin{tabular}{r l } %{\Medium}
    Fachbereich & {\Medium DCSM} \\
    Studiengang & {\Medium Angewandte Informatik}
\end{tabular}%
}
\date{\today}
}

\begin{document}
\maketitle

\begin{frame}{Gliederung}
  \tableofcontents[hideallsubsections]
\end{frame}

\section{Schlüsselergebnisse}
\label{sec:schluesselergebnisse}

\subsection{Was war die Aufgabe?}
\label{subsec:aufg}

\begin{frame}{Was war die Aufgabe?}
  Aufgabe war es, durch die Auswertung von Daten eine gezielte Aussage
  darüber treffen zu können, ob ein Passagier der Titanic das Unglück
  überleben würde oder nicht.

  \bigskip Dafür wurden Daten von
  Kaggle\footnote{\url{https://www.kaggle.com/c/titanic}}
  bereitgestellt, mit denen man versuchen konnte, Modelle
  aufzustellen, die dabei helfen, über das Schicksal der einzelnen
  Passagiere zu entscheiden.
\end{frame}

\subsection{Ansatz}
\label{subsec:ansatz}

\begin{frame}{Ansatz}
  Wir haben uns dafür entschieden, einen \textit{decision tree} aufzustellen.
  Dieser hatte folgende Kriterien:

  \begin{description}
  \item[{\textbf{Das Alter der Passagiers:}}] Hier haben wir unterschieden, ob der
    Passagier noch ein Kind (jünger als 18 Jahre) ist.
  \item[{\textbf{Die Klasse des Passagiers:}}] Die Gäste, die in der ersten Klasse
    mitfuhren, hatten eine bessere Chance die Katastrophe zu überleben.
  \item[{\textbf{Die Anzahl der Geschwister oder Ehepartner:}}] Die Trainingsdaten
    haben gezeigt, dass Alleinreisende eine bessere
    Überlebenschance hatten.
  \end{description}

\end{frame}

\subsection{Ergebnisse}
\label{subsec:erg}

\begin{frame}{Ergebnisse}
  Entgegen unserer Erwartungen hat sich das Ergebnis verschlechtert,
  je mehr Classifier wir hinzugefügt haben.

  \begin{center}
    \begin{tabular}{lrr}
      Classifier                         & Ergebnis & Platz\\
      \hline
      Geschlecht                         & 0.76555 & 3061\\
      Geschlecht, Alter                  & 0.75120 & \\
      Geschlecht, Alter, Klasse          & 0.68900 & \\
      Geschlecht, Alter, Klasse, Familie & 0.68900 & \\
    \end{tabular}
  \end{center}
\end{frame}

\begin{frame}
  \frametitle{tree}
  \Tree [.{Age < 18?} [.{Klasse <\ 3?} 1 [.{Siblings \geq 3?} 0 [.{Age < 10?} 1 0 ] ] ]
                      [.{female?} [.{Klasse < 3?} 1 [.{Siblings > 0?} 0 1 ] ]
                                  [.{Klasse < 2?} 1 0 ] ] ]

                                  Unser decision tree für das letzte
                                  Ergebnis.  Nach links zu gehen
                                  bedeutet, dass die Bedingung des
                                  Knotens erfülllt wurde.  1~steht für
                                  Überleben, eine 0 für den Tod.
\end{frame}

\begin{frame}{Ergebnisse}
  Die Abnahme der Treffer hat uns überrascht, da wir dachten, das
  durch die Verfeinerung der Auswahl die Ergebnisse auch treffender
  werden würden.

  \bigskip Insbesondere bei den männlichen Passagieren schien deren
  Schicksal mehr oder weniger willkürlich. Der Unterschied zwischen
  der Überlebenschance abhängig vom Geschlecht war überraschend, alles
  andere schien im Gegensatz dazu nur kleine Auswirkungen zu haben.

  % \bigskip Dass die Ergebnisse trotzdem immer deutlich über 50\%
  % blieben war gut.
\end{frame}

\section{Offene Fragen}
\label{sec:fragen}

\subsection{Unklarheiten an der Aufgabe}
\label{subsec:unklar}

\begin{frame}{Unklarheiten an der Aufgabe}
  Die Problemstellung war eindeutig.

  \bigskip Einen Ansatz zu entwickeln war ungewohnt, da wenige
  Strategien bisher besprochen wurden und auch bei der
  Implementierungen Unsicherheiten aufkamen.  Darüber hinaus war auch
  das analysieren von Datensätzen ungewohnt und verlief nicht immer
  rund.
\end{frame}

\subsection{Bugs und Unklarheiten im Code}
\label{subsec:code}

\begin{frame}{Bugs und Unklarheiten im Code}
  Was im Code gemacht wird, konnte nachvollzogen werden.
\end{frame}

\subsection{Unplausible Ergebnisse}
\label{subsec:unplausibel}

\begin{frame}{Unplausible Ergebnisse}
  Wie bereits erwähnt, dass sich die Ergebnisse nicht besser geworden
  sind, war überraschend.  Das kann unter Umständen auf Overfitting
  zurückgeführt werden.


\end{frame}

\section{Diskussion}
\label{sec:disku}

\subsection{Alternative Ansätze}
\label{subsec:ansatz}

\begin{frame}{Alternative Ansätze}
  Als alternativen Ansatz haben wir über ein Wertesystem diskutiert,
  bei denen ein Passagier über einen gewissen Schwellwert kommen
  musste, um zu überleben. Bei den Durchläufen wurde uns dann aber
  bewusst, dass es zu wenig Überlebende gab, da die Anforderungen
  zu hoch gesetzt waren.

  \bigskip Auch hatten wir darüber geredet, andere Charakterisitka mit
  einzubeziehen. Diese waren aber entweder zu spezifisch oder nicht
  wirklich aussagekräftig (haben kaum zur Überlebenschance beigetragen).
\end{frame}

\subsection{Würden Sie Ihren Ansatz verändern?}
\label{subsec:veraendern}

\begin{frame}{Würden Sie Ihren Ansatz verändern?}
  Wir würden gerne das Überprüfen der Classifier automatisieren, da
  dieses nicht immer eindeutige Ergebnisse liefert und wir uns auch
  teilweise geirrt hatten, was die Überlebenschancen steigern könnte.
\end{frame}

\subsection{Nächste Schritte}
\label{subsec:next}

\begin{frame}{Nächste Schritte}
  Eventuell maschinelles Lernen einführen, dann wäre es auch möglich,
  ein Wertesystem zu benutzen.
\end{frame}

\end{document}
