\chapter{Generalized Asymmetric Partition Crossover}

\section{Einleitung}
Der „Generalized Asymmetric Partition Crossover“ (GAPX) ist ein
Rekombinationsoperator, der als Teil eines evolutionären Algorithmus
verwendet werden kann.\cite{gapx} Aufgabe einer Rekombination beim ATSP ist es aus
zwei gegebenen Rundreisen $G_1$, $G_2$ ein Kind (auch Offspring genannt)
$G_o$ zu erzeugen.


\begin{algorithm}
\caption{Crossover in einem EA}\label{alg:crossover_ea}
\begin{algorithmic}[1]
\Procedure{Crossover}{$G_1,G_2$}\Comment{}
\State $G_o\gets gapx(G_1,G_2)$\Comment{In 2.2 beschrieben}
\State \textbf{return} $G_o$
\EndProcedure
\end{algorithmic}
\end{algorithm}
GAPX versucht durch ein bestimmtes Verfahren die beiden Rundreisen zu
partitionieren und daraus das neue Kind $G_o$ zu erzeugen.


\section{Funktionsweise}
Seien $G_1$ und $G_2$ beliebige Rundreisen, beide erzeugt aus einem
vollständigen Graphen $G$. Zunächst werden $G_1$ und $G_2$ in einem
gemeinsamen Graphen $G_u$ zusammengeführt. Den Graph $G_u = G_1 \cup
G_2$ nennen wir auch Vereinigungsgraph von $G_1$ und $G_2$.
\newpage
%% Bild mergen von G1, G2
\begin{figure}[hb]
\centering
\renewcommand{\arraystretch}{3.5}
\begin{tabular}{ c c c }
$G_1$ & $G_2$ & $G_u$ \\
\resizebox{90pt}{90pt}{

  \begin{tikzpicture}[%
    >=stealth,
    node distance=2cm,
    on grid,
    auto
  ]
  \node[state] (1){1};
  \node[state] (3) [above right of=1]{3};
  \node[state] (2) [below right of=1]{2};
  \node[state] (4) [below right of=3]{4};
  
  \path[->] (1) edge [blue, bend left=0] node  {} (3);
  \path[->] (3) edge [blue, bend left=0] node  {} (2);
  \path[->] (2) edge [blue, bend left=0] node  {} (4);
  \path[->] (4) edge [blue, bend left=0] node  {} (1);
  
  \end{tikzpicture}
} 

& 

\resizebox{90pt}{90pt}{

  \begin{tikzpicture}[%
    >=stealth,
    node distance=2cm,
    on grid,
    auto
  ]
  \node[state] (1){1};
  \node[state] (3) [above right of=1]{3};
  \node[state] (2) [below right of=1]{2};
  \node[state] (4) [below right of=3]{4};

  \path[->] (1) edge [red, dashed, left=0] node  {} (2);
  \path[->] (2) edge [red, dashed, left=0] node  {} (3);
  \path[->] (3) edge [red, dashed, left=0] node  {} (4);
  \path[->] (4) edge [red, dashed, left=0] node  {} (1);

  \end{tikzpicture}
} 

&

\resizebox{90pt}{90pt}{

  \begin{tikzpicture}[%
    >=stealth,
    node distance=2cm,
    on grid,
    auto
  ]
  \node[state] (1){1};
  \node[state] (3) [above right of=1]{3};
  \node[state] (2) [below right of=1]{2};
  \node[state] (4) [below right of=3]{4};

  \path[->] (1) edge [blue, bend left=0] node  {} (3);
  \path[->] (3) edge [blue, bend left=0] node  {} (2);
  \path[->] (2) edge [blue, bend left=0] node  {} (4);
  \path[->] (4) edge [blue, bend left=0] node  {} (1);

  \path[->] (1) edge [red, dashed, left=0] node  {} (2);
  \path[->] ([xshift=0.4em] 2.north) edge 
      [red, dashed, left=0] node  {} ([xshift=0.4em] 3.south);
  \path[->] (3) edge [red, dashed, left=0] node  {} (4);
  \path[->] ([yshift=0.4em] 4.west) edge 
      [red, dashed, left=0] node  {} ([yshift=0.4em] 1.east);

  \end{tikzpicture}
} 
\end{tabular}
\renewcommand{\arraystretch}{1}
\caption[Beispiel einer Zusammenlegung von Graphen]{
Beispiel für das Zusammenführen der Graphen $G_1$ und $G_2$ zu
dem neuen Graphen $G_u$ anhand zwei Rundreisen mit je vier Knoten.
}
\end{figure}
%%\begin{bem}
%%  Streng genommen ist $G_u$ ein Multigraph, da doppelte Kanten zwischen
%%  Knoten existieren können, jedoch hat dies die Implementierung nicht
%%  eingeschränkt, da ein „normaler Graph“ eine Spezielform eines
%%  Multigraphen ist.
%%\end{bem}
Nach der Zusammenführung der beiden Graphen in $G_u$ muss für alle
Knoten $v$ aus $G_u$ bei der die Bedingung $deg(v) = 4$ erfüllt ist ein
sogenannter Ghost-Knoten $v'$ eingeführt werden. Zusätzlich muss eine
Kante mit dem Gewicht $0$ zwischen $v$ und $v'$ eingefügt werden. 
Der Graph $G_u$ mit eingefügten Ghost-Knoten an den entsprechenden Stellen kennzeichnen wir als $G_u'$.
\begin{figure}[hb]
\centering
\begin{tabular}{ c c }
\resizebox{120pt}{170pt}{
\begin{tikzpicture}[%
>=stealth,
node distance=1.7cm,
on grid
]
\node[state] (1)              {1};


\node[state] (3) [above right of=1]             {3};
\node[state] (2) [below right of=1]            {2};
\node[state] (4) [right of=3]             {4};
\node[state] (5) [right of=2]             {5};

\node[state] (6) [below right of=5] {6};

\node[state] (7) [below left of=6] {7};
\node[state] (10) [left of=7] {10};

\node[state] (11) [below left of=10] {11};

\node[state] (9) [below right of=11] {9};
\node[state] (8) [right of=9] {8};

% solid: parent 1
\path[->] (1) edge [blue, bend left=0] node  {} (3);
\path[->] (3) edge [blue, bend left=0] node  {} (2);
\path[->] (2) edge [blue, bend left=0] node  {} (5);
\path[->] (5) edge [blue, bend left=0] node  {} (4);
\path[->] (4) edge [blue, bend left=0] node  {} (6);
\path[->] (6) edge [blue, bend left=0] node  {} (8);
\path[->] (8) edge [blue, bend left=0] node  {} (7);
\path[->] (7) edge [blue, bend left=0] node  {} (10);
\path[->] (10) edge [blue, bend left=0] node  {} (9);
\path[->] (9) edge [blue, bend left=0] node  {} (11);
\path[->] (11) edge [blue, bend left=0] node  {} (1);

% dashed: parent 2
\path[->] (1) edge [bend left=0, dashed, red] node {} (2);
\path[->] ([xshift=0.7ex] 2.north) edge [red, dashed] node {}
         ([xshift=0.7ex] 3.south);
\path[->] (3) edge [red, dashed] node {}
         (4);
\path[->] ([xshift=0.7ex] 4.south) edge [red, dashed] node {}
         ([xshift=0.7ex] 5.north);
\path[->] (5) edge [red, dashed] node {}
         (6);
\path[->] (6) edge [red, dashed] node {}
         (7);
\path[->] ([xshift=0.7ex] 7.south) edge [red, dashed] node {}
         ([xshift=0.7ex] 8.north);
\path[->] (8) edge [red, dashed] node {}
         (9);
\path[->] ([xshift=0.7ex] 9.north) edge [red, dashed] node {}
         ([xshift=0.7ex] 10.south);
\path[->] (10) edge [red, dashed] node {}
         (11);
\path[->] ([xshift=0.7ex] 11.north) edge [red, dashed] node {}
         ([xshift=0.7ex] 1.south);
\end{tikzpicture}
}

&
\resizebox{120pt}{170pt}{
\begin{tikzpicture}[%
>=stealth,
node distance=1.9cm,
on grid,
auto
]
\node[state] (1){1};
\node[state] (3) [above right of=1]{3};
\node[state] (2) [below right of=1]{2};
\node[state] (4) [right of=3]{4};
\node[state] (5) [right of=2]{5};
\node[state] (6) [above right of=5] {6};
\node[state] (7) [below of=5] {7};
\node[state] (6') [below right of=7] {6'};
\node[state] (10) [left of=7] {10};
\node[state] (11) [below left of=10] {11};
\node[state] (9) [below right of=11] {9};
\node[state] (8) [right of=9] {8};

% solid: parent 1
\path[->] (1) edge [blue, bend left=0] node  {} (3);
\path[->] (3) edge [blue, bend left=0] node  {} (2);
\path[->] (2) edge [blue, bend left=0] node  {} (5);
\path[->] (5) edge [blue, bend left=0] node  {} (4);
\path[->] (6') edge [blue, bend left=0] node  {} (8);
\path[->] (4) edge [blue, bend left=0] node  {} (6);
\path[->] (8) edge [blue, bend left=0] node  {} (7);
\path[->] (7) edge [blue, bend left=0] node  {} (10);
\path[->] (10) edge [blue, bend left=0] node  {} (9);
\path[->] (9) edge [blue, bend left=0] node  {} (11);
\path[->] (11) edge [blue, bend left=0] node  {} (1);
\path[->] (6) edge [blue, bend left=0] node  {} (6');

% dashed: parent 2
\path[->] (1) edge [bend left=0, dashed, red] node {} (2);
\path[->] ([xshift=0.7ex] 2.north) edge [red, dashed] node {}
         ([xshift=0.7ex] 3.south);
\path[->] (3) edge [red, dashed] node {}
         (4);
\path[->] ([xshift=0.7ex] 4.south) edge [red, dashed] node {}
         ([xshift=0.7ex] 5.north);
\path[->] (5) edge [red, dashed] node {}
         (6);
\path[->] (6') edge [red, dashed] node {}
         (7);
\path[->] ([xshift=0.7ex] 7.south) edge [red, dashed] node {}
         ([xshift=0.7ex] 8.north);
\path[->] (8) edge [red, dashed] node {}
         (9);
\path[->] ([xshift=0.7ex] 9.north) edge [red, dashed] node {}
         ([xshift=0.7ex] 10.south);
\path[->] (10) edge [red, dashed] node {}
         (11);
\path[->] ([xshift=0.7ex] 11.north) edge [red, dashed] node {}
         ([xshift=0.7ex] 1.south);
  \path[->] ([xshift=0.7ex] 6.south) edge [red, dashed] node {}
         ([xshift=0.7ex] 6'.north);

\end{tikzpicture}
}
\end{tabular}
\caption[Beispiel Einfügung von Ghost-Knoten]{Links: Ein Graph $G_u$ vor der Einführung von
Ghostknoten. Rechts: Der Graph $G_u'$ mit dem eingefügten Ghostknoten $6'$.
Blaue Linie: $G_1$, erstes Elternteil; rote, gestrichelte Linie: $G_2$,
zweites Elternteil}
\end{figure}
\begin{bem}
Es hat sich herausgestellt, dass es bei der Implementierung einfacher war die
Anzahl der Nachbarn von $v$ anzuschauen, um herauszufinden, ob ein
Ghost-Knoten $v'$ eingefügt werden muss, oder nicht.
\end{bem}
\newpage
Nachdem alle Ghost-Knoten eingefügt wurden müssen Kanten mit einer
bestimmten Eigenschaft aus $G_u'$ entfernt werden, um eine Paritionierung
des Graphen vorzunehmen. In $G_u'$ müssen genau jene Kanten entfernt
werden, die sowohl in $G_1$ \textbf{und} $G_2$ vorkommen. Zusätzlich
müssen die Kanten entfernt werden, die bei der Generierung der
Ghost-Knoten neu eingefügt wurden. In Abbildung 2.2 wäre dies die Kante
zwischen $6$ und $6'$. Die Menge dieser entfernten Kanten nennen wir
nachfolgend $E_c$ und werden auch „Common-Edges“ genannt. Die Menge
beinhaltet also genau die Kanten die in beiden Elternteilen vorkommen. Die
Menge $E_c$ wird nun in die Kantenmenge von $G_o$ (dem Offspring)
eingefügt.
\begin{figure}[hb]
\centering
\resizebox{100pt}{140pt}{
\begin{tikzpicture}[%
>=stealth,
node distance=1.9cm,
on grid,
auto
]
\node[state] (1){1};
\node[state] (3) [above right of=1]{3};
\node[state] (2) [below right of=1]{2};
\node[state] (4) [right of=3]{4};
\node[state] (5) [right of=2]{5};
\node[state] (6) [above right of=5] {6};
\node[state] (7) [below of=5] {7};
\node[state] (6') [below right of=7] {6'};
\node[state] (10) [left of=7] {10};
\node[state] (11) [below left of=10] {11};
\node[state] (9) [below right of=11] {9};
\node[state] (8) [right of=9] {8};

% solid: parent 1
\path[->] (11) edge [bend left=0] node  {} (1);
\path[->] (6) edge [bend left=0] node  {} (6');

\end{tikzpicture}
}
  \caption[Kind $G_o$ nach Einfügen der gemeinsamen Kantenmenge $E_c$]
  {Das Kind $G_o$ im momentanen Zustand. In Abbildung 2.2 sind Ghost-Knoten eingefügt
  worden. $E_c = \{(6,6'),(1,11)\}$. Die Kante $(6,6')$ ist durch das
  Einfügen eines Ghost-Knoten entstanden. Die Kante $(1,11)$ ist in beiden
  Rundreisen $G_1$ und $G_2$ vorhanden. $E_c$ wurde in $G_o$ eingefügt.}
\end{figure}
\newpage
Durch das Entfernen von $E_c$ in $G_u'$ haben wir eine Partitionierung
des Graphen erreicht. Der Graph $G_u'$ wurde in seine Komponenten
zerlegt.
\begin{figure}[bh]
\centering
  \begin{tabular}{c c}
\resizebox{120pt}{170pt}{
\begin{tikzpicture}[%
>=stealth,
node distance=1.9cm,
on grid,
auto
]
\node[state] (1){1};
\node[state] (3) [above right of=1]{3};
\node[state] (2) [below right of=1]{2};
\node[state] (4) [right of=3]{4};
\node[state] (5) [right of=2]{5};
\node[state] (6) [above right of=5] {6};
\node[state] (7) [below of=5] {7};
\node[state] (6') [below right of=7] {6'};
\node[state] (10) [left of=7] {10};
\node[state] (11) [below left of=10] {11};
\node[state] (9) [below right of=11] {9};
\node[state] (8) [right of=9] {8};

% solid: parent 1
\path[->] (1) edge [blue, bend left=0] node  {} (3);
\path[->] (3) edge [blue, bend left=0] node  {} (2);
\path[->] (2) edge [blue, bend left=0] node  {} (5);
\path[->] (5) edge [blue, bend left=0] node  {} (4);
\path[->] (6') edge [blue, bend left=0] node  {} (8);
\path[->] (4) edge [blue, bend left=0] node  {} (6);
\path[->] (8) edge [blue, bend left=0] node  {} (7);
\path[->] (7) edge [blue, bend left=0] node  {} (10);
\path[->] (10) edge [blue, bend left=0] node  {} (9);
\path[->] (9) edge [blue, bend left=0] node  {} (11);
\path[->] (11) edge [blue, bend left=0] node  {} (1);
\path[->] (6) edge [blue, bend left=0] node  {} (6');

% dashed: parent 2
\path[->] (1) edge [bend left=0, dashed, red] node {} (2);
\path[->] ([xshift=0.7ex] 2.north) edge [red, dashed] node {}
         ([xshift=0.7ex] 3.south);
\path[->] (3) edge [red, dashed] node {}
         (4);
\path[->] ([xshift=0.7ex] 4.south) edge [red, dashed] node {}
         ([xshift=0.7ex] 5.north);
\path[->] (5) edge [red, dashed] node {}
         (6);
\path[->] (6') edge [red, dashed] node {}
         (7);
\path[->] ([xshift=0.7ex] 7.south) edge [red, dashed] node {}
         ([xshift=0.7ex] 8.north);
\path[->] (8) edge [red, dashed] node {}
         (9);
\path[->] ([xshift=0.7ex] 9.north) edge [red, dashed] node {}
         ([xshift=0.7ex] 10.south);
\path[->] (10) edge [red, dashed] node {}
         (11);
\path[->] ([xshift=0.7ex] 11.north) edge [red, dashed] node {}
         ([xshift=0.7ex] 1.south);
  \path[->] ([xshift=0.7ex] 6.south) edge [red, dashed] node {}
         ([xshift=0.7ex] 6'.north);

\end{tikzpicture}
}
&
\resizebox{120pt}{170pt}{
\begin{tikzpicture}[%
>=stealth,
node distance=1.9cm,
on grid,
auto
]
\node[state] (1){1};
\node[state] (3) [above right of=1]{3};
\node[state] (2) [below right of=1]{2};
\node[state] (4) [right of=3]{4};
\node[state] (5) [right of=2]{5};
\node[state] (6) [above right of=5] {6};
\node[state] (7) [below of=5] {7};
\node[state] (6') [below right of=7] {6'};
\node[state] (10) [left of=7] {10};
\node[state] (11) [below left of=10] {11};
\node[state] (9) [below right of=11] {9};
\node[state] (8) [right of=9] {8};

% solid: parent 1
\path[->] (1) edge [blue, bend left=0] node  {} (3);
\path[->] (3) edge [blue, bend left=0] node  {} (2);
\path[->] (2) edge [blue, bend left=0] node  {} (5);
\path[->] (5) edge [blue, bend left=0] node  {} (4);
\path[->] (6') edge [blue, bend left=0] node  {} (8);
\path[->] (4) edge [blue, bend left=0] node  {} (6);
\path[->] (8) edge [blue, bend left=0] node  {} (7);
\path[->] (7) edge [blue, bend left=0] node  {} (10);
\path[->] (10) edge [blue, bend left=0] node  {} (9);
\path[->] (9) edge [blue, bend left=0] node  {} (11);

% dashed: parent 2
\path[->] (1) edge [bend left=0, dashed, red] node {} (2);
\path[->] ([xshift=0.7ex] 2.north) edge [red, dashed] node {}
         ([xshift=0.7ex] 3.south);
\path[->] (3) edge [red, dashed] node {}
         (4);
\path[->] ([xshift=0.7ex] 4.south) edge [red, dashed] node {}
         ([xshift=0.7ex] 5.north);
\path[->] (5) edge [red, dashed] node {}
         (6);
\path[->] (6') edge [red, dashed] node {}
         (7);
\path[->] ([xshift=0.7ex] 7.south) edge [red, dashed] node {}
         ([xshift=0.7ex] 8.north);
\path[->] (8) edge [red, dashed] node {}
         (9);
\path[->] ([xshift=0.7ex] 9.north) edge [red, dashed] node {}
         ([xshift=0.7ex] 10.south);
\path[->] (10) edge [red, dashed] node {}
         (11);

\end{tikzpicture}
}

\end{tabular}
\caption[Partitionierung von $G_u'$ nach Löschen von $E_c$]{Der Graph
$G_u'$ nach Löschen der gemeinsamen Kanten $E_c$. Die Teile des
partitionierten Graphen werden Komponenten genannt.}
\end{figure}

Jede Komponente wird als Teilgraph $G_{c_k}$ gekennzeichnet, wobei $k
\leq $Anzahl der Komponenten in $G_u'$. 

\begin{bem}
  In Abbildung 2.4 ist $G_{p_1}$ der Teilgraph mit den Knoten $1,2,3,4,5,6$, $G_{p_2}$ ist der Teilgraph
  mit den Knoten
  $11,9,10,7,8,6'$
\end{bem}
Für alle Komponenten $G_{c_k}$ muss als nächstes die Menge der Eingangs- und
Ausgangsknoten festgestellt werden. Hierfür werden zwei Funktionen
definiert:
\begin{itemize}
  \item $comp_{in}(G_{c_k}, E_c)$ - ermittelt die Menge der
    Eingangsknoten von $G_{c_k}$ 
  \item $comp_{out}(G_{c_k}, E_c)$ - ermittelt die Menge der
    Ausgangsknoten von $G_{c_k}$ 
\end{itemize}
\newpage
\begin{algorithm}
  \caption{Ermittlung Eingangsknoten in $G_{c_k}$}\label{alg:comp_in}
\begin{algorithmic}[1]
  \Procedure{$comp_{in}$}{$G_{c_k}, E_c$}
    \State $V \gets vertices(G_{c_k})$
    \State $Entry \gets list()$
    \Foreach{$v \in V$}
      \State $g = has\_ghostnode\_in\_comp(G_{c_k}, v)$
      \If {$\neg g \land in\_degree(G_{c_k}, v) = 0$}
      \Comment{In-Degree}    
        \State $Entry \gets append(Entry, v)$
      \EndIf
    \EndForeach
    \State \textbf{return} $Entry$
  \EndProcedure
\end{algorithmic}
\end{algorithm}
\begin{algorithm}
\caption{Ermittlung Ausgangsknoten in $G_{c_k}$}\label{alg:comp_out}
\begin{algorithmic}[1]
  \Procedure{$comp_{out}$}{$G_{c_k}, E_c$}
    \State $V \gets vertices(G_{c_k})$
    \State $Exit \gets list()$
    \Foreach{$v \in V$}
      \State $g = has\_ghostnode\_in\_comp(G_{c_k}, v)$
      \If {$\neg g \land out\_degree(G_{c_k}, v) = 0$}
      \Comment{Out-Degree} 
        \State $Exit \gets append(Exit, v)$
      \EndIf
    \EndForeach
    \State \textbf{return} $Exit$
  \EndProcedure
\end{algorithmic}
\end{algorithm}
Die prinzipielle Vorgehensweise der beiden Funktionen ist das Anschauen
der eingehenden beziehungsweise ausgehenden Kanten. Ein Knoten $v$ kommt
nur als Eingangspunkt in Frage, wenn keine eingehenden Kanten für $v$
existieren. Ein ausgehender Knoten kann hingegen keine ausgehende Kanten
besitzen.
\begin{bem}
  Es können Komponenten existieren welche nicht nur einen Eingangs- und
  einen Ausgangsknoten haben. Tatsächlich gibt es sehr häufig
  Partitionierungen mit Komponenten, die drei oder vier Eingängen
  beziehungsweise Ausgängen haben. Die Algorithmen 3 und 4 beachten
  diesen Umstand.
\end{bem}
\begin{bem}
  Die Funktion $has\_ghostnode\_in\_comp(G_{c_k}, v)$ ist hier nicht
  ausführlich erläutert. Sie überprüft, ob für einen Knoten $v$ ein
  entsprechender Ghost-Knoten $v'$ eingefügt wurde, der sich ebenfalls
  innerhalb der Komponente $G_{c_k}$ befindet. Ist dies der Fall, gibt
  die Funktion den Wert true zurück. Knoten, für die ein entsprechender
  Ghost-Knoten innerhalb der Komponente existiert können nicht als
  Eingangs- oder Ausgangspunkt verwendet werden, da sie die Komponente
  in weitere Teile aufspalten würde.
\end{bem}
Mit den Informationen über Eingangs- und Ausgangsknoten ist der „Fluss“ 
einer Komponente bekannt. Man startet in einem Eingangsknoten und endet
in einem Ausgangsknoten. Dieser Fluss innerhalb der Komponente muss nun
näher untersucht werden, da nicht immer eine gültige Partitionierung des
Graphen erfolgt.
Für alle Komponenten $G_{c_k}$ muss überprüft werden,
ob für beide Elternteile ($G_1$, $G_2$) ein Weg vom Eingangsknoten zum
Ausgangsknoten führt. Der Weg muss zudem durch alle Knoten der
Komponente führen. Nur dann ist die Partitionierung gültig. 
Dieser Vorgang ist in Abbildung 2.5 an einem Beispiel demonstriert.

Als letzter Schritt von GAPX muss für jede Komponente $G_{c_k}$ berechnet werden,
ob der Weg von $G_1$, der vom Eingangsknoten zum Ausgangsknoten führt,
günstiger ist als der von $G_2$. Der kürzere Teilweg, beziehungsweise die
Kanten, die diesen Weg repräsentieren, werden in den Offspring $G_o$
übernommen. 

\begin{figure}
\centering
\begin{tabular}{l|c|c}
 & $G_1$ & $G_2$ \\
\hline
\rule{0pt}{15.5ex}
$G_{c_1}$ &

\resizebox{120pt}{80pt}{
\begin{tikzpicture}[%
>=stealth,
node distance=1.9cm,
on grid,
auto
]
\node[state] (1){1};
\node[state] (3) [above right of=1]{3};
\node[state] (2) [below right of=1]{2};
\node[state] (4) [right of=3]{4};
\node[state] (5) [right of=2]{5};
\node[state] (6) [above right of=5] {6};

% solid: parent 1
\path[->] (1) edge [blue, bend left=0] node  {} (3);
\path[->] (3) edge [blue, bend left=0] node  {} (2);
\path[->] (2) edge [blue, bend left=0] node  {} (5);
\path[->] (5) edge [blue, bend left=0] node  {} (4);
\path[->] (4) edge [blue, bend left=0] node  {} (6);

\end{tikzpicture}
}

  &
\resizebox{120pt}{80pt}{
\begin{tikzpicture}[%
>=stealth,
node distance=1.9cm,
on grid,
auto
]
\node[state] (1){1};
\node[state] (3) [above right of=1]{3};
\node[state] (2) [below right of=1]{2};
\node[state] (4) [right of=3]{4};
\node[state] (5) [right of=2]{5};
\node[state] (6) [above right of=5] {6};

% solid: parent 1
\path[->] (1) edge [red, dashed, bend left=0] node  {} (2);
\path[->] (2) edge [red, dashed, bend left=0] node  {} (3);
\path[->] (3) edge [red, dashed, bend left=0] node  {} (4);
\path[->] (4) edge [red, dashed, bend left=0] node  {} (5);
\path[->] (5) edge [red, dashed, bend left=0] node  {} (6);
\end{tikzpicture}
}
\\
\hline
\rule{0pt}{15.5ex}
  $G_{c_2}$ &
\resizebox{120pt}{80pt}{
\begin{tikzpicture}[%
>=stealth,
node distance=1.9cm,
on grid,
auto
]
\node[state] (11){11};
\node[state] (10) [above right of=11]{10};
\node[state] (9) [below right of=11]{9};
\node[state] (7) [right of=10]{7};
\node[state] (8) [right of=9]{8};
\node[state] (6') [above right of=8] {6'};

% solid: parent 1
\path[->] (6') edge [blue, bend left=0] node  {} (8);
\path[->] (8) edge [blue, bend left=0] node  {} (7);
\path[->] (7) edge [blue, bend left=0] node  {} (10);
\path[->] (10) edge [blue, bend left=0] node  {} (9);
\path[->] (9) edge [blue, bend left=0] node  {} (11);
\end{tikzpicture}
}
  &
\resizebox{120pt}{80pt}{
\begin{tikzpicture}[%
>=stealth,
node distance=1.9cm,
on grid,
auto
]
\node[state] (11){11};
\node[state] (10) [above right of=11]{10};
\node[state] (9) [below right of=11]{9};
\node[state] (7) [right of=10]{7};
\node[state] (8) [right of=9]{8};
\node[state] (6') [above right of=8] {6'};

% solid: parent 1
\path[->] (6') edge [red, dashed] node {}
         (7);
\path[->] ([xshift=0.7ex] 7.south) edge [red, dashed] node {}
         ([xshift=0.7ex] 8.north);
\path[->] (8) edge [red, dashed] node {}
         (9);
\path[->] ([xshift=0.7ex] 9.north) edge [red, dashed] node {}
         ([xshift=0.7ex] 10.south);
\path[->] (10) edge [red, dashed] node {}
         (11);
\end{tikzpicture}
}
\end{tabular}
  \caption[Überprüfung, auf eine gültige Partitionierung von
  $G_u'$]{In jeder Partition wird nachgeschaut, ob in den beiden
  Rundreisen $G_1$ und $G_2$ ein Weg vom Eingangsknoten („1“ bzw. „6'“) 
  zum Ausgangsknoten („6“ bzw. „11“) existiert.}
\end{figure}
\newpage
\begin{algorithm}
\caption{Ermittlung kürzerer Teilweg in Komponente $G_{c_k}$} \label{alg:comp_out}
\begin{algorithmic}[1]
  \Procedure{$get\_shorter\_subpath$}{$G_{c_k}, G_1, G_2, E_c$}
    \State $V \gets vertices(G_{c_k})$
    \State $in \gets comp_{in}(G_{c_k}, E_c)$ \Comment{siehe Algorithmus 2} 
    \State $out \gets comp_{out}(G_{c_k}, E_c)$\Comment{siehe Algorithmus 3}
    \State $subpath_{G_1} \gets get\_path(G_1, in, out)$\Comment{von Eingang nach Ausgang}
    \State $subpath_{G_2} \gets get\_path(G_2, in, out)$ 
    \Foreach{$v \in V$}
      \If {$(v \notin subpath_{G_1}) \lor (v \notin subpath_{G_2})$}
        \State \textbf{return} $invalid\_partition$
      \EndIf
    \EndForeach
    \If{$sum\_weights(subpath_{G_1}) < sum\_weights(subpath_{G_2})$}
      \State \textbf{return} $subpath_{G_1}$
    \Else
      \State \textbf{return} $subpath_{G_2}$
    \EndIf
  \EndProcedure
\end{algorithmic}
\end{algorithm}
Die kürzesten Teilwege für alle Komponenten (entweder von $G_1$ oder von
$G_2$ stammend fügen wir dem Offspring $G_o$ hinzu. Ein möglicher
Offspring könnte nun beispielsweise wie in Abbildung 2.6 aussehen.
\begin{figure}[bh]
  \centering
\resizebox{100pt}{140pt}{
\begin{tikzpicture}[%
>=stealth,
node distance=1.9cm,
on grid,
auto
]
\node[state] (1){1};
\node[state] (3) [above right of=1]{3};
\node[state] (2) [below right of=1]{2};
\node[state] (4) [right of=3]{4};
\node[state] (5) [right of=2]{5};
\node[state] (6) [above right of=5] {6};
\node[state] (7) [below of=5] {7};
\node[state] (6') [below right of=7] {6'};
\node[state] (10) [left of=7] {10};
\node[state] (11) [below left of=10] {11};
\node[state] (9) [below right of=11] {9};
\node[state] (8) [right of=9] {8};

% solid: parent 1
\path[->] (11) edge [dashed, bend left=0] node  {} (1);
\path[->] (6) edge [dashed, bend left=0] node  {} (6');

\path[->] (6') edge [] node {}
         (7);
\path[->] ([xshift=0.7ex] 7.south) edge [] node {}
         ([xshift=0.7ex] 8.north);
\path[->] (8) edge [] node {}
         (9);
\path[->] ([xshift=0.7ex] 9.north) edge [] node {}
         ([xshift=0.7ex] 10.south);
\path[->] (10) edge [] node {}
         (11);
\path[->] (1) edge [bend left=0] node  {} (3);
\path[->] (3) edge [bend left=0] node  {} (2);
\path[->] (2) edge [bend left=0] node  {} (5);
\path[->] (5) edge [bend left=0] node  {} (4);
\path[->] (4) edge [bend left=0] node  {} (6);
\end{tikzpicture}
}
  \caption[Finaler Offspring $G_o$ erzeugt aus $G_1$ und $G_2$]{Der
  finale Zustand des Kindes $G_o$ nach Einfügen der Kantenmenge $E_c$
  und der kürzesten Teilwege der Komponenten. Gestrichelte Linien: Menge
  $E_c$, normale Linien: kürzeste Teilwege}
\end{figure}
%%\parbox[b][3cm][t]{4cm}{
%  \begin{tabular}{l r}
%    Komponente: & ${G_{c_1}}$ \\
%    Eingangsknoten: & $\{1\}$ \\
%    Ausgangsknoten: & $\{6\}$ \\
%  \end{tabular}
%}
