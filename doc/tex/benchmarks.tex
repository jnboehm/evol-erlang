\chapter{Ergebnisse}
Bei den ATSP-Instanzen der TSPLIB~\cite{tsplib}, bei denen die Knotenanzahl $\leq 33$ 
beträgt, wurden die bekannten, optimalen Lösungen gefunden. Bei größeren
Problemen konnten die optimalen Lösungen nicht gefunden werden. Die
nachfolgende Tabelle zeigt eine Zusammenfassung der Ergebnisse. Sofern
keine optimale Lösung gefunden wurde, haben wir den Vorgang manuell
abgebrochen, wenn keine wesentliche Verbesserung der Fitness ersichtlich
war.
\begin{table}[H]
  \centering
{
  \setmainfont[Numbers={Uppercase,Monospaced}]{Vollkorn}
  \begin{tabular}{lrrr}
    Problem & Beste Lösung & Erreichte Fitness & Generationen \\
    \hline
    br17    & 39           & 39                & 1            \\
    ftv33   & 1286         & 1286              & 199          \\
    ftv35   & 1473         & 1475              & 196          \\
    ftv38   & 1530         & 1536              & 286          \\
    ftv44   & 1613         & 1653              & 237          \\
    ftv47   & 1776         & 1794              & 233          \\
    ftv55   & 1608         & 1691              & 252          \\
    ry48p   & 14422        & 14608             & 313          \\
  \end{tabular}
}
  \caption[Ergebnisse TSPLIB Instanzen]{Die Ergebnisse einiger asymmetrischen Instanzen der
  TSPLIB~\cite{tsplib}}
\end{table}
\begin{figure}[H]
    \centering
    \begin{subfigure}[b]{0.45\textwidth}
      \includegraphics[width=\textwidth]{../research/scores/br17-300-24}
      %\caption{br17-300-24}
      \caption{br17}
      \begin{center}
        \vskip-10pt
        \small 300 Generationen, 24 Prozesse
      \end{center}
      \label{fig:br17-300-24}
    \end{subfigure}
    ~
    % add desired spacing between images, e. g. ~, \quad, \qquad, \hfill etc.
    %(or a blank line to force the subfigure onto a new line)
    \begin{subfigure}[b]{0.45\textwidth}
      \includegraphics[width=\textwidth]{../research/scores/ftv33-300-12}
      %\caption{ftv33-300-12}
      \caption{ftv33}
      \begin{center}
        \vskip-10pt
        \small 300 Generationen, 12 Prozesse
      \end{center}
      \label{fig:ftv33-300-12}
    \end{subfigure}

    \begin{subfigure}[b]{0.45\textwidth}
      \includegraphics[width=\textwidth]{../research/scores/ftv44-300-12}
      %\caption{ftv44-300-12}
      \caption{ftv44}
      \begin{center}
        \vskip-10pt
        \small 300 Generationen, 12 Prozesse
      \end{center}
      \label{fig:ftv44-300-12}
    \end{subfigure}
    ~
    \begin{subfigure}[b]{0.45\textwidth}
      \includegraphics[width=\textwidth]{../research/scores/ry48p-300-12}
      \caption{ry48p}
      \begin{center}
      \vskip-10pt
      \small 300 Generationen, 12 Prozesse
      \end{center}
      \label{fig:ry48p-300-12}
    \end{subfigure}
    \caption[Fitnesswerte einiger ATSP-Probleme]{\label{fig:fitness-plots} 
    Fitnesswerte über die ersten 100 Generationen. Die Graphen sind
    normiert, die besten, bekannten Lösungen befinden sich bei $1.0$. Schwarz:
    Durchschnittlicher Fitness-Wert im Verlauf des EA, Rot: Bester
    Fitness-Wert pro Generation}
\end{figure}
\noindent
Nach der Implementierung unseres Programms nahmen wir Kontakt mit den
Autoren von \cite{gapx} auf, um herauszufinden, nach wie vielen
Generationen ihre Anwendung terminierte. Als Antwort wurde uns der
Quellcode von GAPX samt dem dazugehörigen EA bereitgestellt. Im
Vergleich benötigt unsere Implementierung bei allen Problemen (bis auf
br17) mehr Generationen oder findet nicht die optimale Lösung. Der Grund
dafür kann an der nicht durchgeführten „Immigration“ des EA liegen.
Wenn wir diese implementiert hätten, kämen für unser Programm nur kleine
Probleme in Frage.
