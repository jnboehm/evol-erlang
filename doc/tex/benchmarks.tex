\chapter{Ergebnisse}
Unsere Ergebnisse wurden mit denen der Authoren verglichen.  Dabei
sind unsere Lösungen konsistent schlechter.

{
  \setmainfont[Numbers={Uppercase,Monospaced}]{Vollkorn}
  \begin{tabular}{lrrr}
    Problem & Beste Lösung & Erreichte Fitness & Generationen \\
    \hline
    br17    & 39           & 39                & 1            \\
    ftv33   & 1286         & 1296              & 199          \\
    ftv35   & 1473         & 1475              & 196          \\
    ftv38   & 1530         & 1536              & 286          \\
    ftv44   & 1613         & 1653              & 237          \\
    ftv47   & 1776         & 1794              & 233          \\
    ftv55   & 1608         & 1691              & 252          \\
    ry48p   & 14422        & 14608             & 313          \\
  \end{tabular}
}

\begin{figure}[ht]
    \centering
    \begin{subfigure}[b]{0.45\textwidth}
      \includegraphics[width=\textwidth]{../research/scores/ry48p-300-12}
      \caption{ry48p-300-12}
      \label{fig:ry48p-300-12}
    \end{subfigure}
    ~ %add desired spacing between images, e. g. ~, \quad, \qquad, \hfill etc.
      %(or a blank line to force the subfigure onto a new line)
    \begin{subfigure}[b]{0.45\textwidth}
      \includegraphics[width=\textwidth]{../research/scores/br17-300-24}
      \caption{br17-300-24}
      \label{fig:br17-300-24}
    \end{subfigure}

    % add desired spacing between images, e. g. ~, \quad, \qquad, \hfill etc.
    %(or a blank line to force the subfigure onto a new line)
    \begin{subfigure}[b]{0.45\textwidth}
      \includegraphics[width=\textwidth]{../research/scores/ftv33-300-12}
      \caption{ftv33-300-12}
      \label{fig:ftv33-300-12}
    \end{subfigure}
    ~
    \begin{subfigure}[b]{0.45\textwidth}
      \includegraphics[width=\textwidth]{../research/scores/ftv44-300-12}
      \caption{ftv44-300-12}
      \label{fig:ftv44-300-12}
    \end{subfigure}
    \caption{\label{fig:fitness-plots} Fitnesswerte über die ersten
      100 Generationen}
\end{figure}
