\chapter{Evolutionärer Algorithmus}
\section{Einleitung}
Ein „evolutionärer Algorithmus“ (kurz EA) ist ein Algorithmus, welcher 
versucht ein Optimierungsproblem mithilfe der in der Biologie 
vorkommenden Gesetze der Evolution zu lösen. Ziel des Algorithmus ist es
ein Ergebnis über mehrere Generationen zu verbessern. Für das Traveling-
Salesman-Problem bedeutet dies, dass versucht wird aus einer Menge
von zufällig generierten Rundreisen eine möglichst kurze Rundreise zu
finden.

\section{Basis}
\begin{tikzpicture}[scale=1,align=center]
  \def \r {\textwidth/3}
  \def \margin {4}
  \node (rekombination) at (0:\r) {Rekombination};
  \node (paarungsselektion) at (60:\r) {Paarungs-\\selektion};
  \node (terminierungsbedingung) at (120:\r) {Terminierungs-\\bedingung};
  \node (umweltselektion) at (180:\r) {Umwelt-\\selektion};
  \node (bewertung) at (240:\r) {Bewertung};
  \node (mutation) at (300:\r) {Mutation};

  \node (init) at (145:\r *1.75) {Init};
  \path[draw, ->, >=latex] (init) .. controls (165:\r *1.75) and (165:\r) ..
                  (terminierungsbedingung.south west);

  \draw[<-, >=latex] (4:\r) arc (4:50:\r);
  \draw[<-, >=latex] (305:\r) arc (305:356:\r);
  \draw[<-, >=latex] (246:\r) arc (246:292:\r);
  \draw[<-, >=latex] (186:\r) arc (186:235:\r);
  \draw[<-, >=latex] (130:\r) arc (130:175:\r);
  \draw[<-, >=latex] (75:\r) arc (75:100:\r);

  %% this goes form node to node in y cycle but because of the points
  %% where it leaves each node it makes the image loook wonky
  %
  % \path[draw] (rekombination) .. controls (20:5cm) and (40:5cm) ..
  % (paarungsselektion) .. controls (80:5cm) and (100:5cm) ..
  % (terminierungsbedienung) .. controls (140:5cm) and (160:5cm) ..
  % (umweltselektion) .. controls (200:5cm) and (220:5cm) ..
  % (bewertung) .. controls (260:5cm) and (280:5cm) ..
  % (mutation) .. controls (320:5cm) and (340:5cm) ..
  % (rekombination);

\end{tikzpicture}

\section{Initialisierung}
In der Initialisierungsphase des EA werden $n$ Rundreisen aus einem
vollständigen Graphen zunächst zufällig erzeugt. Direkt im Anschluss
wird das in Kapitel 3 vorgestellte Verfahren „ls3opt“ angewendet, um die
Rundreise an lokalen Stellen zu verbessern.
\section{Terminierungsbedingung}
Der EA wird beendet, wenn $1500$ Generationen durchlaufen wurden, oder
die bekannte optimale Lösung erreicht wurde.
\section{Paarungsselektion}
\section{Rekombination}
\section{Mutation}
\section{Bewertung}
\section{Umweltselektion}
