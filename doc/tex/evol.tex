\chapter{Evolutionärer Algorithmus}
\section{Einleitung}
Ein „evolutionärer Algorithmus“ (kurz EA) ist ein Algorithmus, welcher 
versucht ein Optimierungsproblem mithilfe der in der Biologie 
vorkommenden Gesetze der Evolution zu lösen. Ziel des Algorithmus ist es
ein Ergebnis über mehrere Generationen zu verbessern. Für das Traveling-
Salesman-Problem bedeutet dies, dass versucht wird aus einer Menge
von zufällig generierten Rundreisen eine möglichst kurze Rundreise zu
finden.

\section{Basis}

\begin{figure}[H]
\begin{tikzpicture}[scale=1,align=center]
  \def \r {\textwidth/3}
  \def \margin {4}
  \node (rekombination) at (0:\r) {Rekombination};
  \node (paarungsselektion) at (60:\r) {Paarungs-\\selektion};
  \node (terminierungsbedingung) at (120:\r) {Terminierungs-\\bedingung};
  \node (umweltselektion) at (180:\r) {Umwelt-\\selektion};
  \node (bewertung) at (240:\r) {Bewertung};
  \node (mutation) at (300:\r) {Mutation};

  \node (init) at (145:\r *1.75) {Init};
  \path[draw, ->, >=latex] (init) .. controls (165:\r *1.75) and (165:\r) ..
                  (terminierungsbedingung.south west);

  \draw[<-, >=latex] (4:\r) arc (4:50:\r);
  \draw[<-, >=latex] (305:\r) arc (305:356:\r);
  \draw[<-, >=latex] (246:\r) arc (246:292:\r);
  \draw[<-, >=latex] (186:\r) arc (186:235:\r);
  \draw[<-, >=latex] (130:\r) arc (130:175:\r);
  \draw[<-, >=latex] (75:\r) arc (75:100:\r);

  %% this goes form node to node in y cycle but because of the points
  %% where it leaves each node it makes the image loook wonky
  %
  % \path[draw] (rekombination) .. controls (20:5cm) and (40:5cm) ..
  % (paarungsselektion) .. controls (80:5cm) and (100:5cm) ..
  % (terminierungsbedienung) .. controls (140:5cm) and (160:5cm) ..
  % (umweltselektion) .. controls (200:5cm) and (220:5cm) ..
  % (bewertung) .. controls (260:5cm) and (280:5cm) ..
  % (mutation) .. controls (320:5cm) and (340:5cm) ..
  % (rekombination);

\end{tikzpicture}
\caption[Standardablauf eines evolutionären Algorithmus]{Der
  Standardablauf eines evolutionären Ablaufs}
\end{figure}
Der EA, der in diesem Projekt verwendet wurde basiert auf einer
Arbeit von Yuichi Nagata und David Soler und weicht an manchen
Stellen von diesem Standard ab~\cite{nagata}. Die entsprechenden
Paramter des EA sind in der Ausarbeitung zu GAPX von Tinós et al. 
enthalten~\cite{gapx}.

\begin{algorithm}
\caption{EA nach Nagata und Soler }\label{alg:ea}
\begin{algorithmic}[1]
  \Procedure{$EA$}{$GenLimit$}
    \State $P \gets pop\_init()$\Comment{Initiale Generation}
    \While{Abbruchbedingung nicht erfüllt}
      \State $(G_1, G_2) \ gets selection(P)$
      \State $G_o \gets crossover(G_1, G_2)$
      \If{Crossover hat Lösung nicht verbessert}
        \State $G_o \gets mutation(G_o)$
      \EndIf
      \State $P \gets append(P, G_o)$
      \If{Beste Lösung nach 20. Gen nicht verbessert}
        \State Nachbarschaftsgröße für ls3opt erhöhen
      \EndIf
      \State $P = next\_generation(P)$
    \EndWhile
    \State \textbf{return} $best\_solution$
  \EndProcedure
\end{algorithmic}
\end{algorithm}
\section{Initialisierung}
In der Initialisierungsphase des EA werden $n$ Rundreisen aus einem
vollständigen Graphen zunächst zufällig erzeugt. Direkt im Anschluss
wird das in Kapitel 3 vorgestellte Verfahren „ls3opt“ angewendet, um die
Rundreise an lokalen Stellen zu verbessern. Die initiale
\textit{Nachbarschaftsgröße für „ls3opt“ beträgt 10}. Sie wird um 1
inkrementiert, falls in den letzten 20 Generationen die beste Lösung
nicht verbessert wurde.
\section{Terminierungsbedingung}
Der EA wird beendet, wenn \textit{1500 Generationen} durchlaufen wurden, oder
die bekannte optimale Lösung erreicht wurde. 
\section{Paarungsselektion}
Zur Selektion einer Rundreise wird eine \textit{Turnierselektion} angewendet, 
bei der zwei zufällige Rundreisen ausgewählt werden~\cite{weicker}. Jene mit dem
besseren Fitness-Wert wird mit einer \textit{Wahrscheinlichkeit von 0.8} als 
Elternteil verwendet~\cite{gapx}. Die Selektion findet also zwei mal 
(für $G_1$ und für $G_2$) statt. Sind die beiden Elternteile $G_1$ und
$G_2$ bestimmt kann die Rekombination mittels GAPX durchgeführt werden.
\section{Rekombination}
Als Rekombinationsoperator wurde GAPX verwendet, der in Kapitel 2
ausführlich beschrieben wurde. Zwei Rundreisen müssen vorher ausgewählt
werden. Anschließend kann der Crossover, wie in
\hyperref[alg:crossover_ea]{Algorithmus 1}
beschrieben angewandt werden.
\section{Mutation}
Die Mutation wird bei allen Rundreisen durchgeführt, die einen
schlechteren Fitness-Wert haben als die derzeit beste Rundreise. 
Bei der Mutation wird eine zufällige Anzahl von „ls2opt“-Vorgänge
(Anzahl $\leq 10$) durchgeführt. Nach der Modifikation der Rundreise wird
zusätzlich noch ein „ls3opt“-Vorgang, wie in \hyperref[ls3opt_func]{Kapitel
3.2} erläutert, durchgeführt.

\section{Bewertung}
Die Bewertung der Rundreise findet mittels einer Fitness-Funktion statt.
Zur Ermittlung der Kantengewichte in einem Grpahen $G = (V, E)$ wird
eine Funktion $d(e)$, mit $e \in E$ definiert, die für eine Kante das
entsprechende Gewicht zurückgibt.
\begin{align*}
   f(G) = \sum_{e \in E} d(e)
\end{align*}
Die Summe der Kantengewichte entspricht der Fitness des Graphen von
$G$. Um so kleiner der Wert von $f(G)$, um so besser ist die Fitness des
Graphen.

\section{Umweltselektion}
\textit{Elitismus} wird angewandt,
um die beste Rundreise immer in die nächsten Generation zu übernehmen.
Der Fitness-Wert während des evolutionären Algorithmus kann sich während
des Prozesses somit niemals verschlechtern. Die beste Lösung wird
zurückgegeben, wenn die Abbruchbedingung erreicht wurde.
\section{Parameterübersicht}
\begin{tabular}{l|p{5.4cm}}
  \textbf{Parameter} & \textbf{Wert} \\
  \hline
  Maximum Generationen & 1500 \\
  \hline
  Nachbarschaftsgröße „ls3opt“ & 10 \\
  \hline
  Erhöhung Nachbarschaftsgröße „ls3opt“ & 1 \\
  \hline
  Paarungsselektion & Turnierselektion, \newline mit Wahrscheinlichkeit 0.8 \\
  \hline
  Rekombination & GAPX\\
  \hline
  Mutation & ls3opt, ls2opt\\
  \hline
  Umweltselektion & Elitismus \\
  \hline
\end{tabular}
