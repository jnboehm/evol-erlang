\chapter{Lokale Suche mit 3-opt}
\section{Einleitung}
Ein weiterer Teil des Projektes war die Implementierung der lokalen
Suche namens „3-opt“ oder auch „ls3opt“ genannt. Dieses lokale 
Suchverfahren wurde zum Erstellen der initialen Population des 
evolutionären Algorithmus verwendet. Ebenfalls wurde „ls3opt“ als Teil
des Mutationsoperators verwendet. Ziel des Verfahrens ist es durch das 
Löschen und anschließende Neuverbinden von drei Kanten eine Verbesserung der Rundreise zu erzielen. Die Implementierung basiert auf der in
\cite{nagata} vorgestellten, modifizierten Vorgehensweise. Die Vorgehensweise „reduziert die Rechenzeit“\cite{gapx} durch
eine sogenannte „Don't-look-bits“-Strategie. Basis dieses Verfahrens ist
eine Ausarbeitung von Kanellakis und Papadimitriou\cite{ls3opt_atsp} aus dem Jahre 1979,
die auch zur Einarbeitung benutzt wurde. 

\section{Funktionsweise}
Sei $G = (V,E)$ eine Rundreise, erzeugt aus einem vollständigen Graphen,
mit $\#V \geq 3$.
Seien $v_1$, $v_3$, $v_5$ beliebige Knoten aus der Rundreise $G$, mit
der Bedingung, dass es einen Weg von $v_1$ nach $v_3$ gibt und ebenfalls
einen Weg von $v_3$ nach $v_5$. Dadurch ist sichergestellt, dass die
Reihenfolge der Knoten in der Rundreise wie folgt lautet: $v_1, \dotsc, v_3, \dotsc, v_5$
\begin{figure}[bh]
\centering
\resizebox{330pt}{30pt}{

  \begin{tikzpicture}[%
    >=stealth,
    node distance=2cm,
    on grid,
    auto
  ]
  \node[state] (1){$v_1$};
  \node[state] (2) [right of=1]{$v_2$};
  \node[state] (3) [right of=2]{$v_3$};
  \node[state] (4) [right of=3]{$v_4$};
  \node[state] (5) [right of=4]{$v_5$};
  \node[state] (6) [right of=5]{$v_6$};

  \path[->] (1) edge [left=0] node  {} (2);
  \path[->] (2) edge [left=0] node  {} (3);
  \path[->] (3) edge [left=0] node  {} (4);
  \path[->] (4) edge [left=0] node  {} (5);
  \path[->] (4) edge [left=0] node  {} (5);
  \path[->] (5) edge [left=0] node  {} (6);

  \end{tikzpicture}
}
\caption[Ausgangssituation 3-opt]{Die Ausgangssituation des
3-opt-Verfahren. Die Abbildung zeigt nur einen Teil einer größeren
Rundreise.}
\end {figure}

\begin{figure}[bh]
\centering
\resizebox{330pt}{30pt}{

  \begin{tikzpicture}[%
    >=stealth,
    node distance=2cm,
    on grid,
    auto
  ]
  \node[state] (1){$v_1$};
  \node[state] (2) [right of=1]{$v_2$};
  \node[state] (3) [right of=2]{$v_3$};
  \node[state] (4) [right of=3]{$v_4$};
  \node[state] (5) [right of=4]{$v_5$};
  \node[state] (6) [right of=5]{$v_6$};

  \path[->] (2) edge [left=0] node  {} (3);
  \path[->] (4) edge [left=0] node  {} (5);

  \end{tikzpicture}
}
\caption[Löschen von drei Kanten in 3-opt]{Es wurden die Kanten
$(v_1,v_2)$,$(v_3,v_4)$ und $(v_5,v_6)$ gelöscht.}
\end{figure}
Zunächst werden die ausgehenden Kanten der Knoten $v_1$, 
$v_3$ und $v_5$ gelöscht. Dies kann man in Abbildung 3.2 beispielhaft
erkennen. Nachdem nun drei Kanten gelöscht wurden, müssen die Kanten wie
folgt neu verbunden werden, damit die Richtung der Rundreise beibehalten
wird.\cite{nagata}: 
\begin{align*}
  (v_1, v_4)\\
  (v_5, v_2)\\
  (v_3, v_6)
\end{align*}
Nachdem die Kanten neu verbunden wurden kann mittels
Aufsummieren der Kantengewichte herausgefunden werden, ob die Rundreise
verbessert wurde. Das Löschen und neu verbinden von drei Kanten wird
auch „3-opt-move“ genannt.


\begin{figure}[bh]
\centering
\resizebox{330pt}{90pt}{

  \begin{tikzpicture}[%
    >=stealth,
    node distance=2cm,
    on grid,
    auto
  ]
  \node[state] (1){$v_1$};
  \node[state] (2) [right of=1]{$v_2$};
  \node[state] (3) [right of=2]{$v_3$};
  \node[state] (4) [right of=3]{$v_4$};
  \node[state] (5) [right of=4]{$v_5$};
  \node[state] (6) [right of=5]{$v_6$};

  \path[->] (1) edge [bend left=40] node  {} (4);
  \path[->] (5) edge [bend left=40] node  {} (2);
  \path[->] (3) edge [bend right=40] node  {} (6);
  \path[->] (2) edge [left=0] node  {} (3);
  \path[->] (4) edge [left=0] node  {} (5);

  \end{tikzpicture}
}
\caption[Neues Verbinden der Kanten in 3-opt]{Die neuen Kanten wurden
eingefügt, die Richtung der Rundreise wurde beibehalten.}
\end{figure}

\section{„Don't look bits“-Strategie}
Der „3-opt-move“, wie in  Kapitel 3.2 beschrieben, beschreibt das Löschen
und Neuverbinden von drei Kanten innerhalb einer Rundreise, wenn drei
Kanten gegeben sind. Die „Don't look bits“-Strategie beschäftigt sich
mit dem Thema, wie eine Rundreise mit $n$ Knoten in ihrer Gänze durch
sehr viele „3-opt-move“-Vorgänge verbessert werden kann.

\begin{figure}[bh]
  \centering
  \begin{tikzpicture}
  \edef\turingtapesize{0.5cm}
\tikzstyle{tape}=[draw,minimum size=\turingtapesize]

% Drawing the tape itself
\begin{scope}[start chain=0 going right,node distance=0mm]
    \node[on chain=0,tape,draw=none](ü)     {$\ldots$};
    \node[on chain=0,tape]          (a)     {$v_1$};
    \node[on chain=0,tape]          (b)     {$v_2$};
    \node[on chain=0,tape]          (c)     {$v_3$};
    \node[on chain=0,tape]          (d)     {$v_4$};
    \node[on chain=0,tape]          (e)     {$v_5$};
    \node[on chain=0,tape]          (f)     {$v_6$};
    \node[on chain=0,tape]          (g)     {$v_7$};
    \node[on chain=0,tape]          (h)     {$v_8$};
    \node[on chain=0,tape]          (i)     {$v_9$};
    \node[on chain=0,tape]          (l)     {$v_{10}$};
    \node[on chain=0,tape]          (l)     {$v_{11}$};
    \node[on chain=0,tape]          (l)     {$v_{12}$};
    \node[on chain=0,tape]          (m)     {$v_{13}$};
    \node[on chain=0,tape]          (n)     {$v_{14}$};
    \node[on chain=0,tape]          (o)     {$v_{15}$};
    \node[on chain=0,tape]          (p)     {$v_{16}$};
    \node[on chain=0,tape,draw=none](u)     {$\ldots$};

\end{scope}
\end{tikzpicture}
  \caption[asd]{asd}
\end{figure}
\section{Ergebnisse}
