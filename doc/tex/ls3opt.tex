\chapter{Optimierung mit 3-opt}
\section{Einleitung}
Ein weiterer Teil des Projektes war die Implementierung der
Graphen-Optimierung namens „3-opt“ oder auch „ls3opt“ genannt. Dieses lokale 
Optimierungsverfahren wurde zum Erstellen der initialen Population des 
evolutionären Algorithmus verwendet. Ebenfalls wurde „ls3opt“ als Teil
des Mutationsoperators verwendet. Ziel des Verfahrens ist es durch 
Löschen und anschließendes Neuverbinden von drei Kanten eine Verbesserung der Rundreise zu erzielen. Die Implementierung basiert auf der in
\cite{nagata} vorgestellten, modifizierten Vorgehensweise. Das Verfahren „reduziert die Rechenzeit“\cite{gapx} durch
eine sogenannte „Don't-look-bits“-Strategie. Basis dieses Verfahrens ist
eine Ausarbeitung von Kanellakis und Papadimitriou\cite{ls3opt_atsp} aus dem Jahre 1979,
die auch zur Einarbeitung benutzt wurde. 

\section{Funktionsweise}
\label{ls3opt_func}
Sei $G = (V,E)$ eine Rundreise, erzeugt aus einem vollständigen Graphen,
mit $\#V \geq 3$.
Seien $v_1$, $v_3$, $v_5$ beliebige Knoten aus der Rundreise $G$, mit
der Bedingung, dass es einen Weg von $v_1$ nach $v_3$ gibt und ebenfalls
einen Weg von $v_3$ nach $v_5$. Dadurch ist sichergestellt, dass die
Reihenfolge der Knoten in der Rundreise wie folgt lautet: $v_1, \dotsc, v_3, \dotsc, v_5$
\begin{figure}[bh]
\centering
\resizebox{330pt}{30pt}{

  \begin{tikzpicture}[%
    >=stealth,
    node distance=2cm,
    on grid,
    auto
  ]
  \node[state] (1){$v_1$};
  \node[state] (2) [right of=1]{$v_2$};
  \node[state] (3) [right of=2]{$v_3$};
  \node[state] (4) [right of=3]{$v_4$};
  \node[state] (5) [right of=4]{$v_5$};
  \node[state] (6) [right of=5]{$v_6$};

  \path[->] (1) edge [left=0] node  {} (2);
  \path[->] (2) edge [left=0] node  {} (3);
  \path[->] (3) edge [left=0] node  {} (4);
  \path[->] (4) edge [left=0] node  {} (5);
  \path[->] (4) edge [left=0] node  {} (5);
  \path[->] (5) edge [left=0] node  {} (6);

  \end{tikzpicture}
}
\caption[Ausgangssituation 3-opt]{Die Ausgangssituation des
3-opt-Verfahren. Die Abbildung zeigt nur einen Teil einer größeren,
  beispielhaften Rundreise.}
\end {figure}

\begin{figure}[bh]
\label{fig:3opt_2}
\centering
\resizebox{330pt}{30pt}{

  \begin{tikzpicture}[%
    >=stealth,
    node distance=2cm,
    on grid,
    auto
  ]
  \node[state] (1){$v_1$};
  \node[state] (2) [right of=1]{$v_2$};
  \node[state] (3) [right of=2]{$v_3$};
  \node[state] (4) [right of=3]{$v_4$};
  \node[state] (5) [right of=4]{$v_5$};
  \node[state] (6) [right of=5]{$v_6$};

  \path[->] (2) edge [left=0] node  {} (3);
  \path[->] (4) edge [left=0] node  {} (5);

  \end{tikzpicture}
}
\caption[Löschen von drei Kanten in 3-opt]{Es wurden die Kanten
$(v_1,v_2)$,$(v_3,v_4)$ und $(v_5,v_6)$ gelöscht.}
\end{figure}
\noindent
Zunächst werden die ausgehenden Kanten der Knoten $v_1$, 
  $v_3$ und $v_5$ gelöscht. Dies kann man in
  \hyperref[fig:3opt_2]{Abbildung 3.2} beispielhaft
erkennen. Nachdem nun drei Kanten gelöscht wurden, müssen die Kanten wie
folgt neu verbunden werden, „damit die Richtung der Rundreise beibehalten
wird“\cite{nagata}, $v_1$ bleibt Eingangs- und $v_6$ bleibt
Ausgangsknoten: 
\begin{align*}
  (v_1, v_4)\\
  (v_5, v_2)\\
  (v_3, v_6)
\end{align*}
Nun kann mittels
Aufsummieren der Kantengewichte herausgefunden werden, ob die Rundreise
verbessert wurde. Das Löschen und Neuverbinden von drei Kanten wird
auch „3-opt-move“ genannt.


\begin{figure}[bh]
\centering
\resizebox{330pt}{90pt}{

  \begin{tikzpicture}[%
    >=stealth,
    node distance=2cm,
    on grid,
    auto
  ]
  \node[state] (1){$v_1$};
  \node[state] (2) [right of=1]{$v_2$};
  \node[state] (3) [right of=2]{$v_3$};
  \node[state] (4) [right of=3]{$v_4$};
  \node[state] (5) [right of=4]{$v_5$};
  \node[state] (6) [right of=5]{$v_6$};

  \path[->] (1) edge [bend left=40] node  {} (4);
  \path[->] (5) edge [bend left=40] node  {} (2);
  \path[->] (3) edge [bend right=40] node  {} (6);
  \path[->] (2) edge [left=0] node  {} (3);
  \path[->] (4) edge [left=0] node  {} (5);

  \end{tikzpicture}
}
\caption[Neues Verbinden der Kanten in 3-opt]{Die neuen Kanten wurden
eingefügt, die Richtung der Rundreise wurde beibehalten.}
\end{figure}

\section{„Don't look bits“-Strategie}
Der „3-opt-move“, wie in  \hyperref[ls3opt_func]{Kapitel 3.2} beschrieben, beschreibt das Löschen
und Neuverbinden innerhalb einer Rundreise, wenn drei
Kanten gegeben sind. Die „Don't look bits“-Strategie beschäftigt sich
mit dem Thema, wie eine Rundreise mit $n$ Knoten in ihrer Gänze durch
sehr viele „3-opt-move“-Vorgänge verbessert werden kann.

Hierzu wird zunächst für jeden Knoten $v$ aus einem Graphen
eine Nachbarschaft gebildet. Eine Nachbarschaft $\mathcal{N}(v)$ mit der
Bedigung $v = v_1$ nennen wir Nachbarschaft von $v$. Im initialen Fall 
darf $v_4$ nur $10$ Knoten von $v_1$
entfernt sein. Somit ist sichergestellt, dass „3-opt-move“-Vorgänge nur
innerhalb einer begrenzten Nachbarschaft ausgeführt werden und somit die
Komplexität des Verfahrens verringert wird.

\begin{bem}
  Bei der Implementierung wurde der Einfachheit halberüberprüft, 
  ob $v_5$ $10$ Knoten von $v_1$ entfernt ist (nicht $v_4$). Dies hat
  die Implementierung vereinfacht. 
\end{bem}
\begin{figure}[bh]
  \centering
  \begin{tikzpicture}
  \edef\turingtapesize{0.5cm}
\tikzstyle{tape}=[draw,minimum size=\turingtapesize]

% Drawing the tape itself
\begin{scope}[start chain=0 going right,node distance=0mm]
    \node[on chain=0,tape,draw=none](ü)     {$\ldots$};
    \node[on chain=0,tape]          (a)     {$v'_1$};
    \node[on chain=0,tape]          (b)     {$v'_2$};
    \node[on chain=0,tape]          (c)     {$v'_3$};
    \node[on chain=0,tape, red]          (d)     {$v'_4$};
    \node[on chain=0,tape, red]          (e)     {$v'_5$};
    \node[on chain=0,tape, red]          (f)     {$v'_6$};
    \node[on chain=0,tape, red]          (g)     {$v'_7$};
    \node[on chain=0,tape, red]          (h)     {$v'_8$};
    \node[on chain=0,tape, red]          (i)     {$v'_9$};
    \node[on chain=0,tape, red]          (l)     {$v'_{10}$};
    \node[on chain=0,tape, red]          (l)     {$v'_{11}$};
    \node[on chain=0,tape, red]          (l)     {$v'_{12}$};
    \node[on chain=0,tape, red]          (m)     {$v'_{13}$};
    \node[on chain=0,tape, red]          (n)     {$v'_{14}$};
    \node[on chain=0,tape]          (o)     {$v'_{15}$};
    \node[on chain=0,tape]          (p)     {$v'_{16}$};
    \node[on chain=0,tape,draw=none](u)     {$\ldots$};

\end{scope}
\end{tikzpicture}
  \caption[Beispiel einer Nachbarschaft $\mathcal{N}(v)$]{Beispiel
  einer Nachbarschaft. Rot: Nachbarschaft von $v'_4$, Größe der
  Nachbarschaft: $10$.}
\end{figure}
\noindent
Es wird nun versucht innerhalb aller Nachbarschaften der Rundreise 3-Tupel zu finden, (die
Knoten für den „3-opt-move“ $v_1$, $v_3$ und $v_5$) welche die Rundreise
verbessern.
\noindent
Falls eine Verbesserung gefunden wurde, muss für die nachfolgenden
3-Tupel die neue Rundreise benutzt werden. Wenn die Rundreise jedoch
nicht verbessert wurde, speichern wir uns die drei Werte aus dem 3-Tupel
in einer Liste, der sogenannten „Don't look bit“-Liste. Diese Werte dürfen in den
nachfolgenden Nachbarschaften nicht mehr verwendet werden. Der
Algorithmus terminiert, wenn alle Knoten aus der Rundreise in der
„Don't-look-bit“-Liste vorhanden sind und somit keine lokalen
Optimierungen mehr vorgenommen werden können.
\newpage

\begin{algorithm}
  \caption{„3-opt-moves“ für alle Nachbarschaften von Rundreise $G$}\label{alg:ls3opt_run}
\begin{algorithmic}[1]
  \Procedure{$ls3opt\_run$}{$G, NSize$} \Comment{NSize = Nachbarschaftsgröße}
    \State $V \gets vertices(G)$
    \State $DontLook \gets list()$
    \Foreach{$v \in V$}
      \State $N \gets get\_neighborhood(v, DontLook, NSize)$\Comment{siehe Bem. 8}
      \State $C \gets valid\_3\_tuple(N, DontLook)$\Comment{siehe Bem. 9}
      \Foreach{$c \in C$}
        \State $v_1 \gets c[0]$
        \State $v_3 \gets c[1]$
        \State $v_5 \gets c[2]$
        \State $G' \gets optmove3(G, v_1,v_3,v_5)$ \Comment{siehe Kap. \hyperref[ls3opt_func]{3.2}}
        \If {$sum\_weights(G') < sum\_weights(G)$}
          \State $G \gets G'$ \Comment{Rundreise wurde verbessert}
        \Else
          \State $DontLook \gets append(DontLook, v_1)$ 
          \State $DontLook \gets append(DontLook, v_3)$
          \State $DontLook \gets append(DontLook, v_5)$
        \EndIf
      \EndForeach
      \If{$V = DontLook$}
        \State \textbf{return} $G$ \Comment{Optimierte Rundreise}
      \EndIf
    \EndForeach
    \State \textbf{return} $G$
  \EndProcedure
\end{algorithmic}
\end{algorithm}

\begin{bem}
  Die Funktion $get\_neighborhood(v, DontLook, NSize)$ gibt für den
  Knoten $v$ und der Nachbarschaftsgröße $NSize$ die entsprechende
  Nachbarschaft zurück. Die Nachbarschaft, die zurückgegeben wird, darf
  dabei keine Werte enthalten, die in $DontLook$ vorhanden sind.
\end{bem}

\begin{bem}
Die Funktion $valid\_3\_tuple(N, DontLook)$ gibt anhand der
  Nachbarschaft $N$ alle 3-Tupel zurück, die für „3-opt-move“-Vorgänge
  in Frage kommen. Die Tupel dürfen keine Werte enthalten, die in
  $DontLook$ vorhanden sind.
\end{bem}
\noindent
In Zeile 15,16 und 17 des \hyperref[alg:ls3opt_run]{Algorithmus 5} wurden die drei Knoten, die keine
Verbesserung ergeben haben, in die „Don't-look-bit“-Liste aufgenommen.
Dadurch wird mit diesen Knoten nicht mehr versucht Verbesserungen in der
Rundreise zu finden. 
\section{Laufzeitverhalten}
Im Rahmen des Projektes wurde zusätzlich noch das Laufzeitverhalten des
in Erlang implementierten Verfahrens untersucht. 
\begin{figure}[H]
  \centering
  \includegraphics[width=\textwidth]{../research/ls3opt}
  \caption[Laufzeit von ls3opt im Verhältnis zu der Anzahl der
  Knoten]{\label{fig:ls3optcomplxty} Die Laufzeit von ls3opt im
    Verhältnis zu der Anzahl der Knoten in dem Graphen (der Anzahl der
    Städte in einer Rundreise).   In unserer Implementierung beträgt $a$ ungefähr
    $6,\kern-1pt 3113 \times 10^{-4}$. Die Ausgleichsfunktion wurde mit
    dem Least-Squares-Verfahren ermittelt.} % fix kerning after comma in math mode
  %% erwähnen, wie wir auf den wert gekommen sind? Least squares?
\end{figure}
\noindent
Die Laufzeit der naiven Implementierung ist kubisch, wie bei
dessen ersten Einführung in \cite{lin1965computer} beschrieben
wurde. Die Laufzeit wurde von $\mathcal{O}(n^3)$ auf
$\mathcal{O}(n^{2.2})$ verbessert, da die „Don't-Look-bits“-Strategie
verwendet wurde.
