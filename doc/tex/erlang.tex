\chapter{Erlang}
Erlang ist eine dynamische und funktionale Programmiersprache.  Die
Sprache wurde im „Ericsson Computer Science Laboratory“ für große
Telekommunikationsanwendungen entwickelt, wo viel Wert auf
Nachrichtenaustausch und Distribution gelegt wird.

Obwohl funktionale Programmiersprachen generell frei von
Seiteneffekten sind, werden einzelne Prinzipien gebrochen, wenn
praktische Probleme dies erfordern.  Eine Datenbanktabelle kann zum
Beispiel verändert werden, ohne gleich eine komplette Kopie der
Datenbank zu erstellen.

Bei dem Nachrichtenaustausch (Message-passing) versenden die einzelnen
Prozesse Informationen mittels Nachrichten.  Hierbei gibt es keinen
gemeinsamen Speicher, was das Message-passing sicher macht – das
Versenden einer Nachricht kann nie fehlschlagen.  Da wir den
Algorithmus durch Message-passing parallelisieren wollten, haben wir
uns für Erlang als Programmiersprache entschieden.

Erlangs Prozesse benötigen initial deutlich weniger Speicher als
Systemprozesse, da sie diesen dynamisch zugewiesen bekommen.  Die
Nebenläufigkeit wird in Erlang mittels Threads realisiert, deren
Scheduling von der Laufzeitumgebung selber betrieben wird.

Ein Erlang-Prozess entspricht ungefähr einer Node.Die Node
erscheint im Betriebssystem als ein Prozess und kann
wiederum mehrere Prozesse erzeugen, die von ihr selber verwaltet
werden.  Nach einmaliger Verbindung mit einer anderen Node können
diese über Message-passing kommunizieren, ähnlich zu der Kommunikation
innerhalb einer Node.  Für mehr Details dazu siehe
\cite[Kapitel~„Distribunomicon“]{lyse}.

\section{Vorteile}
Unsere Wahl fiel auf Erlang, da diese Sprache für distributierte und
skaliebare Systeme konzipiert wurde.

Das Versenden und Empfangen von Nachrichten und damit das Verteilen
von Informationen ist durch das Pattern-matching von Erlang leicht zu
realisieren.  In dem Modul \lstinline!digraph!  werden Graphen intern
durch drei Datenbanktabellen repräsentiert, auf die jeder Prozess in
einer Node zugreifen kann, weshalb nur die Referenz auf die Tabellen
versendet werden muss, wenn ein Graph innerhalb einer Node verschickt
wird.

Auch kann man durch Nachrichten ein Interface für den Benutzer
erstellen, in dem man an den Hauptprozess eine Nachricht schickt und
er darauf antwortet.

\medskip
\renewcommand{\figurename}{Listing}
\begin{figure}[ht]
  \centering
  \begin{lstlisting}
  fac(0) -> 1;
  fac(N) when is_integer(N) andalso N > 0 ->
      N * fac(N-1).
    \end{lstlisting}
    \caption{\label{lst:fac} Funktion zur Berechnung der Fakultät einer Zahl}
\end{figure}
\renewcommand{\figurename}{Abbildung}


Algorithmen können aufgrund der Ähnlichkeit zur mathematischen Notation
in Erlang leicht umgesetzt werden. Hier helfen vor allem Header-guards und
Higher-order-functions.

Header-guards sind Ausdrücke in Funktionsparametern, die einen
bestimmten Wert haben müssen oder mithilfe von einzelnen Funktionen
getestet werden können, wie in dem Listing \ref{lst:fac} zu sehen ist.
Diese ersetzen zum großen Teil \lstinline!if!-Abfragen.

Higher-order-functions ersetzen zum großen Teil die
\lstinline!for!-Schleife von imperativen Sprachen.  Solche Funktionen
bekommen eine Liste und eine Funktion und wenden diese (meistens) auf
jedes Element der Liste an.  Dabei kann eine neue Liste entstehen oder
es werden bestimmte Werte akkumuliert.  Diese Art von Funktionen
erlauben einfache und saubere Transformationen, die in vielen Fällen
die Absicht der Anweisung übersichtlich ausdrücken.

Darüber hinaus hilft auch die Unveränderbarkeit von Variablen. Sobald
eine Variable einmalig einen Wert zugewiesen bekommt, kann sich dieser
nicht mehr ändern, was den Zustand einer
Variable zur Laufzeit eindeutig werden lässt.

\section{Nachteile}
\label{sec:disadv}
Für rechenintensive Aufgaben ist Erlang ungeeignet.  Dynamische
Sprachen haben generell das Problem, dass sie Typinformationen zur
Laufzeit haben müssen und der Kompiler deshalb weniger Optimierungen
vornehmen kann.  Es führt außerdem dazu, dass der Typ zur Laufzeit
überprüft werden muss, was ebenfalls Rechenzeit kostet.

Generell wird davon abgeraten, die Sprache für rechenintensive Aufgaben zu
benutzen, da Erlang dafür nicht konzipiert wurde
\cite[Kapitel~3]{lyse}.  Stattdessen soll man Erlang für Kommunikation
verwenden und für rechenintensive Aufgaben auf Sprachen
zurückzufallen, die schnelleren Maschinencode produzieren und nicht
auf einer virutellen Maschine laufen.

\section{Die Funktion \gtwght}
\label{sec:get-weight}

In unserer Implementierung gibt die Funktion \gtwght\ – die Notation
bedeutet, dass die Funktion drei Parameter übergeben bekommt – das
Gewicht zwischen zwei Knoten in dem übergebenen Graphen zurück.  Der Profiler
\lstinline!eprof!  zeigte, dass über 90\% der Aufrufe, und damit auch
ein Großteil der Zeit, dieser Funktion zufallen.  Deshalb hatten wir
versucht \gtwght\ zu optimieren, da die Anzahl der Aufrufe nicht
reduziert werden konnte, denn in ls3opt wurde diese Information
benötigt.  Beim Vergleich mit dem Code von den Autoren von \cite{gapx}
wurden wir in der Hinsicht bestätigt. % soll das erwähnt werden?

Zunächst wurde die Funktion so implementiert, dass \gtwght\ den
Graphen selbst übergeben bekommt und dann auf dessen Tabellen eine
Datenbankabfrage ausführt.  Obwohl die In-Memory-Datenbank, in Erlang
wird diese „Erlang Term Storage“ (ETS) genannt, einen konstanten
Zeitaufwand für Lese- und Schreiboperationen hat, war die konstante
Zugriffszeit zu hoch.  Deshalb wurde vorher eine Liste mit allen
Kanten erstellt aus denen mit Hilfe von Pattern-matching der gefundene
Wert zurückgegeben wurde.  

Die Laufzeit verbrachte weiterhin einen Großteil der Zeit in \gtwght.
Daher wollten wir durch eine „Native Implemented Function“ (NIF) in C
die Zeit pro Aufruf weiter verringern.  Das Resultat hat die Zeit pro
Aufruf leider verschlechtert, da die Makros für die Datenkonvertierung
(von Erlangs dynamischen Termen zu C-Typen) zu viel Zeit beanspruchen.

Die Funktion \gtwght\ benötigt also immer noch die meisten Zeit in
unserem Programm.  Dieser Teil wäre in einer Sprache wie C schnell und
intuitiv gewesen, da lediglich ein Zugriff auf ein zweidimensionales
Array an den übergebenen Indizes nötig gewesen wäre.

Es gibt in Erlang Arrays, diese sind allerdings „immutable“ – müssen
also bei einer Veränderung komplett neu erstellt werden – und bieten
vor allem keinen wahlfreien Speicherzugriff \cite[Kapitel~11]{lyse}.
Arrays hätten uns bei dem Auslesen von Gewichtswerten ebenfalls nicht die
nötige Geschwindigkeit geliefert. % komisches wort
