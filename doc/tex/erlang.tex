\chapter{Erlang}
Da wir den Algorithmus durch „message-passing“ parallelisieren wollten
haben wir uns für Erlang als unsere Programmiersprache verwendet.  Die
Sprache wurde im „Ericsson Computer Science Laboratory“ entwickelt um
vor allem Telekommunikationsanwendungen zu entwickeln, wo viel Wert
auf Nachrichtenaustausch und Distribution gelegt wird.

Erlang ist eine funktionale Programmiersprache, obwohl einzelne
Prinzipien gebrochen werden, wenn praktische Probleme dies gebieten.
Zum Beispiel wird referenzielle Transparenz – „Ein Ausdruck kann mit
seinem Ergebnis ersetzt werden, ohne die Korrektheit des Programms zu
ändern“ – gebrochen, damit das Ergebnis von \texttt{now()} überhaupt
sinnvoll sein kann.

\section{Vorteile}
Erlang hat sich hervorragend geeignet, das Programm nebenläufig zu
machen.  Das war auch der Hauptgrund, warum wir uns für diese Sprache
entschieden haben.
\section{Nachteile}
Für rechenintensive Aufgaben ist Elang schlicht ungeeignet.
Dynamische Sprachen haben generell das Problem, dass sie
Typinformationen zur Laufzeit haben müssen und der Kompiler deshalb
weniger Optimierungen vornehmen kann.

Generell wird davon abgeraten, die Sprache für solche Zwecke zu
benutzen, da Erlang dafür nicht konzipiert wurde.
\cite[Kapitel~3]{lyse}
