\chapter{Einführung}
\section{„Asymmetrisches Traveling Salesman“ (ATSP) Problem}
Das Problem des Handlungsreisenden ist ein
\textbf{NP-hartes}\cite{exakte_algo} kombinatorisches Problem, bei dem
versucht wird, in
einem vollständigen, \textit{gerichteten} Graphen die kürzeste Rundreise zu
finden. Jeder Knoten muss dabei genau einmal besucht werden.

\section{Komplexität}
Um die kürzeste Rundreise in einem vollständigen, gerichteten Graphen zu
ermitteln, kann man die Kosten für jede mögliche Rundreise berechnen,
um den optimalen Weg zu finden. Es gibt $(n-1)!$
Rundreisen, die zu untersuchen sind\cite{pursuit}. Dies bedeutet, dass 
der Rechenaufwand für das Problem 
des Handlungsreisenden exponentiell mit der Anzahl der Knoten im Graphen
wächst, dadurch $\in 
\mathcal{O}(n!)$ und somit auch~$\in$~\textbf{NP}, falls 
\textbf{P} $\neq$ \textbf{NP}.
\begin{bem}
Die Anzahl der möglichen Rundreisen bei einer Knotenanzahl von $4$ wären also
$(4-1)! = 1 \cdot 2 \cdot 3 = 6$. Wählen wir $n = 20$ würde die
Anzahl der möglichen Rundreisen bereits $(20-1)! =
121645100408832000$ betragen.
\end{bem}

\section{Lösungsidee}
In der Praxis werden heutzutage trotz allem „gute“ Lösungen benötigt.
Eine direkte Berechnung der besten Lösung ist aber für eine große
Knotenanzahl nicht praxistauglich. In dieser Ausarbeitung wird
vorgestellt, wie man gute Lösungen mit einem evolutionären
Algorithmus heuristisch berechnen kann. Ebenfalls wird eine in der
Programmiersprache Erlang entwickelte Implementierung des Algorithmus 
beschrieben.

In der Ausarbeitung wird beschrieben wie der Rekombinationsoperator
„Generalized Asymmetric Partition Crossover“ als Teil des evolutionären
Prozesses funktioniert und implementiert wurde.
Ein weiterer notwendiger Teil der Implementierung ist die Optimierung
von Rundreisen mit einem modifiziertem „3-opt“-Verfahren.
