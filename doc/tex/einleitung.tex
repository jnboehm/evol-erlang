\chapter{Einführung}
\section{„Asymmetrisches Traveling Salesman“ (ATSP) Problem}
Das Problem des Handlungsreisenden ist ein
\textbf{NP-hartes} kombinatorisches Problem, bei dem versucht wird in
einem vollständigen gerichtetem Graphen die kürzeste Rundreise zu finden. Jeder
Knoten darf dabei nur einmal besucht werden. Der Zusatz „asymmetrisch“
bedeutet, dass eine Kante zwischen $v_j$ und $v_k$ ein anderes Gewicht
hat als die Kante zwischen $v_k$ und $v_j$. $v_j$ und $v_k$ sind
dabei beliebige Knoten aus dem vollständigen Graphen.

\section{Komplexität}
\begin{theorem}
Die Komplexität zum Herausfinden der kürzesten Rundreise in einem
vollständigen Graphen ist $\mathcal{O}(n!)$. 
\end{theorem}

\begin{proof}[Beweis]
Sei $G=(V,E)$, wobei $V = \{ v_0, v_1, \dotsc, v_n\}$, $E= ?$ ein
vollständiger gerichteter Graph, das bedeutet, von einem
beliebigen Knoten $v_j$ mit $j \leq n$ sind alle anderen Knoten $v_0, v_1, \dotsc, v_n$
erreichbar. Wir starten eine beliebige Rundreise bei dem Knoten $v_j$.
Da wir $v_j$ nicht mehr besuchen dürfen, bleiben uns noch 
$n-1$ Knoten zur Auswahl, die wir von $v_j$ besuchen können. Für den
darauffolgende Knoten bleiben uns $n-2$ Knoten, und so weiter. Dies bedeutet, dass 
die Anzahl der möglichen Touren $(n-1)!$ beträgt.
\begin{align*}
  (n-1)! = (n-1) \cdot (n-2) \cdot \dotsc \cdot 2 \cdot 1
\end{align*}
Daraus folgt, dass das Problem des Handlungsreisenden $\subseteq \mathcal{O}(n!)$
\end{proof}
\begin{bem}
Die Anzahl der möglichen Rundreisen bei einer Knotenanzahl von $4$ wären also
$(4-1)! = 1 \cdot 2 \cdot 3 = 6$. Wählen wir $n = 20$ würde die
Anzahl der möglichen Rundreisen bereits $(20-1)! =
121645100408832000$ betragen.
\end{bem}

\section{Lösungsidee}
In der Praxis werden heutzutage trotz allem „gute“ Lösungen benötigt.
Eine direkte Berechnung der besten Lösung ist aber für eine große
Knotenanzahl nicht praxistauglich. In dieser Ausarbeitung wird
vorgestellt, wie man gute Lösungen mit einem sogenannten „evolutionären
Algorithmus“ heuristisch findet. Ebenfalls wird eine in der
Programmiersprache Erlang entwickelte Implementierung des Algorithmus 
vorgestellt.

In der Ausarbeitung wird beschrieben wie der Rekombinationsoperator
„Generalized Asymmetric Partition Crossover“ funktioniert und
implementiert wurde.
Ein weiterer notwendiger Teil der Implementierung ist die lokale Suche
mit einem modifiziertem 3-opt-Verfahren.
