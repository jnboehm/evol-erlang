\documentclass[compress]{beamer}

% basically only used for the \to in \item
\usepackage[math]{kurier}
\usepackage{listings}
\usepackage[ngerman]{babel}
\usepackage{qtree}
\usepackage{fontspec}
\usepackage{xcolor}
\usepackage{tikz}
\usetikzlibrary{snakes,arrows,shapes}

\usepackage[noflama]{hsrm}

\lstset{%
  basicstyle=\small\ttfamily, %
  commentstyle=\color{gray}, %
  stringstyle=\color{Green}, %
  keywordstyle=\bfseries\color{hsrmRed}, %
  frame=tb, % line at Top and one at Bottom
  showstringspaces=false, %
  language=Java, %
  moredelim=[is][\textcolor{gray}]{\%\%}{\%\%}, %colorize opt params
  moredelim=[is][\textcolor{hsrmSec2Dark}]{==}{==}, %To overcome the weird
  % behavior of Strings in
  % beamer
}

\newcommand{\hsrm}{Hochschule {\Medium RheinMain}}

\AtBeginDocument{
  \title{Traveling Salesman Problem}
  \subtitle{Evolutionäre Algorithmen in Erlang}
  \author{Niklas Böhm \& Jens Nazarenus}
  \institute{%
    \begin{tabular}{r l } %{\Medium}
      Fachbereich & {\Medium DCSM} \\
      Studiengang & {\Medium Angewandte Informatik}
    \end{tabular}%
  }
  \date{\today}
}

\begin{document}
\maketitle

\begin{frame}{Gliederung}
  \tableofcontents[hideallsubsections]
\end{frame}

\section{Traveling Salesman Problem}
\label{sec:tsp}

\frame{\frametitle{Problemvorstellung} 
\begin{itemize}
  \item Gegeben sei ein Graph $G$
  \item Ziel: Finde eine möglichst kurze \glqq Rundreise\grqq{}
  \item Bedingung 1: Besuche jeden Knoten aus $G$ nur einmal
  \item Bedingung 2: Startknoten $=$ Endknoten
\end{itemize}
  
\begin{tikzpicture}[>=latex,line join=bevel,]
%%
\begin{scope}
  \pgfsetstrokecolor{black}
  \definecolor{strokecol}{rgb}{1.0,1.0,1.0};
  \pgfsetstrokecolor{strokecol}
  \definecolor{fillcol}{rgb}{1.0,1.0,1.0};
  \pgfsetfillcolor{fillcol}
\end{scope}
  \node (A) at (10.0bp,9.5bp) [draw,ellipse] {A};
  \node (C) at (157.0bp,55.5bp) [draw,ellipse] {C};
  \node (B) at (83.0bp,51.5bp) [draw,ellipse] {B};
  \node (E) at (304.5bp,9.5bp) [draw,ellipse] {E};
  \node (D) at (231.5bp,55.5bp) [draw,ellipse] {D};
  \draw [->] (D) ..controls (251.74bp,43.033bp) and (273.02bp,29.249bp)  .. (E);
  \definecolor{strokecol}{rgb}{0.0,0.0,0.0};
  \pgfsetstrokecolor{strokecol}
  \draw (268.5bp,46.5bp) node {1};
  \draw [->] (A) ..controls (29.9bp,20.68bp) and (50.243bp,32.714bp)  .. (B);
  \draw (46.5bp,44.5bp) node {4};
  \draw [->] (C) ..controls (178.01bp,55.5bp) and (196.31bp,55.5bp)  .. (D);
  \draw (194.5bp,64.5bp) node {3};
  \draw [->] (B) ..controls (103.99bp,52.61bp) and (122.45bp,53.636bp)  .. (C);
  \draw (119.5bp,62.5bp) node {2};
  \draw [->] (E) ..controls (281.36bp,9.5bp) and (254.93bp,9.5bp)  .. (232.5bp,9.5bp) .. controls (82.0bp,9.5bp) and (82.0bp,9.5bp)  .. (82.0bp,9.5bp) .. controls (64.493bp,9.5bp) and (44.557bp,9.5bp)  .. (A);
  \draw (157.0bp,18.5bp) node {12};
%
\end{tikzpicture}

 
}

\frame{\frametitle{Abgrenzung TSP-Varianten}
\begin{itemize}
  \item Traveling Salesman Problem (TSP)
  \item Asymetric TSP
  \item metrisches TSP
\end{itemize}

}


\frame{\frametitle{Komplexität}
  \begin{itemize}
    \item kombinatorisches Problem 
    \item \textbf{NP-hard}
  \end{itemize}
}

\frame{\frametitle{Lösungsmöglichkeiten}

}



\section{Evolutionäre Algorithmen}
\label{sec:evolutionäre}
\frame{\frametitle{Grundlegende Begriffe}
  \begin{itemize}
    \item Optimierungsproblem
  \end{itemize}
}
\frame{\frametitle{Evolutionärer Ablauf}

}

\frame{\frametitle{Initialisierung}

}

\frame{\frametitle{Abbruchbedingung}

}

\frame{\frametitle{Paarungsselektion}

}

\frame{\frametitle{Rekombination}

}

\frame{\frametitle{GAPX}

}

\frame{\frametitle{Mutation}

}

\frame{\frametitle{Bewertung mittels Fitness-Funktion}

}

\frame{\frametitle{Selektion}

}

\frame{\frametitle{Coming up}

}

\section{Erlang vorstellen}
\label{sec:erlang}

\section{Parallelisierung}
\label{sec:parallelisierung}

\section{Auswertung der (bisherigen) Ergebnisse}
\label{sec:auswertung}

\end{document}
