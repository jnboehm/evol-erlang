\documentclass[compress]{beamer}

% basically only used for the \to in \item
\usepackage[math]{kurier}
\usepackage{listings}
\usepackage[ngerman]{babel}
\usepackage{qtree}
\usepackage{fontspec}
\usepackage{xcolor}
\usepackage{tikz}
\usepackage{graphicx}
\usepackage{graphviz}
\usepackage{auto-pst-pdf}
\usetikzlibrary{snakes,arrows,shapes}

\usepackage[noflama]{hsrm}


\lstset{%
  basicstyle=\small\ttfamily, %
  commentstyle=\color{gray}, %
  stringstyle=\color{Green}, %
  keywordstyle=\bfseries\color{hsrmRed}, %
  frame=tb, % line at Top and one at Bottom
  showstringspaces=false, %
  language=Java, %
  moredelim=[is][\textcolor{gray}]{\%\%}{\%\%}, %colorize opt params
  moredelim=[is][\textcolor{hsrmSec2Dark}]{==}{==}, %To overcome the weird
  % behavior of Strings in
  % beamer
}

\newcommand{\hsrm}{Hochschule {\Medium RheinMain}}

\AtBeginDocument{
  \title{Traveling Salesman Problem}
  \subtitle{Evolutionäre Algorithmen in Erlang}
  \author{Niklas Böhm \& Jens Nazarenus}
  \institute{%
    \begin{tabular}{r l } %{\Medium}
      Fachbereich & {\Medium DCSM} \\
      Studiengang & {\Medium Angewandte Informatik}
    \end{tabular}%
  }
  \date{\today}
}

\begin{document}
\maketitle

\begin{frame}{Gliederung}
  \tableofcontents[hideallsubsections]
\end{frame}

\section{Traveling Salesman Problem}
\label{sec:tsp}

\frame{\frametitle{Problemvorstellung} 
\begin{itemize}
  \item Gegeben sei ein Graph $G$
  \pause
  \item Ziel: Finde eine möglichst kurze „Rundreise“
  \pause
  \item Bedingung 1: Besuche jeden Knoten aus $G$ nur einmal  
  \pause
  \item Bedingung 2: Startknoten $=$ Endknoten
  \pause
\end{itemize}
\vfill
\vfill
\digraph[scale=0.6]{prob001}{  
  rankdir=LR; 
  A->B[label="2"]; 
  B->C[label="2"];
  C->D[label="4"];
  D->E[label="2"];
  E->A[label="9"];
}
\vfill
}

\frame{\frametitle{Abgrenzung TSP-Varianten}
\begin{itemize}
  \item Traveling Salesman Problem (TSP)
  \pause
  \item Asymmetric TSP
  \pause
  \item metrisches TSP
\end{itemize}
}

\frame{\frametitle{Asymmetric TSP}
  \begin{itemize}
    \item Kantengewichte zwischen zwei Knoten unterschiedlich
  \end{itemize}
  \vfill
  \digraph[scale=0.6]{prob002}{  
    rankdir=LR;
    A->B[label="9"];
    B->A[label="5"];
  }
  \vfill
}

\frame{\frametitle{Komplexität}
  \begin{itemize}
    \item kombinatorisches Problem 
    \item \textbf{NP-hard}
  \end{itemize}
}

\frame{\frametitle{Lösungsmöglichkeiten}
}



\section{Evolutionäre Algorithmen}
\label{sec:evolutionäre}
\frame{\frametitle{Grundlegende Begriffe}
  \begin{itemize}
    \item Optimierungsproblem
  \end{itemize}
}
\frame{\frametitle{Evolutionärer Ablauf}

}

\frame{\frametitle{Initialisierung}

}

\frame{\frametitle{Abbruchbedingung}

}

\frame{\frametitle{Paarungsselektion}

}

\frame{\frametitle{Rekombination}

}

\frame{\frametitle{GAPX}

}

\frame{\frametitle{Mutation}

}

\frame{\frametitle{Bewertung mittels Fitness-Funktion}

}

\frame{\frametitle{Selektion}

}

\frame{\frametitle{Coming up}

}

\section{Erlang vorstellen}
\label{sec:erlang}

\section{Parallelisierung}
\label{sec:parallelisierung}

\section{Auswertung der (bisherigen) Ergebnisse}
\label{sec:auswertung}

\end{document}
