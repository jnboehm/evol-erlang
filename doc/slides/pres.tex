\documentclass[compress]{beamer}

% basically only used for the \to in \item
\usepackage[math]{kurier}
\usepackage{listings}
\usepackage[ngerman]{babel}
\usepackage{qtree}
\usepackage{fontspec}
\usepackage{xcolor}
\usepackage{tikz}
\usepackage{graphicx}
\usepackage{graphviz}
\usepackage{auto-pst-pdf}
\usepackage{tikz}
\usetikzlibrary{shapes,shapes.geometric,arrows,fit,calc,positioning,automata}

\def\mathunderline#1#2{\color{#1}\underline{{\color{hsrmWarmGreyDark}#2}}\color{hsrmWarmGreyDark}}

% verschiedene Shapes für die Zustände
\tikzset{elliptic state/.style={draw,ellipse}}
\tikzset{rect state/.style={draw,rectangle}}

\usepackage[noflama]{hsrm}

\lstset{%
  basicstyle=\small\ttfamily, %
  commentstyle=\color{gray}, %
  stringstyle=\color{Green}, %
  keywordstyle=\bfseries\color{hsrmRed}, %
  frame=tb, % line at Top and one at Bottom
  showstringspaces=false, %
  language=erlang, %
  moredelim=[is][\textcolor{gray}]{\%\%}{\%\%}, %colorize opt params
  moredelim=[is][\textcolor{hsrmSec2Dark}]{==}{==}, %To overcome the weird
  % behavior of Strings in
  % beamer
}

\newcommand{\hsrm}{Hochschule {\Medium RheinMain}}

\AtBeginDocument{
  \title{Traveling Salesman Problem}
  \subtitle{Evolutionäre Algorithmen in Erlang}
  \author{Niklas Böhm \& Jens Nazarenus}
  \institute{%
    \begin{tabular}{r l } %{\Medium}
      Fachbereich & {\Medium DCSM} \\
      Studiengang & {\Medium Angewandte Informatik}
    \end{tabular}%
  }
  \date{\today}
}

\begin{document}
\maketitle

\begin{frame}{Gliederung}
  \tableofcontents[hideallsubsections]
\end{frame}

\section{Traveling Salesman Problem}
\label{sec:tsp}

\frame{\frametitle{Problemvorstellung}
\begin{itemize}
  \item Gegeben sei ein gerichteter Graph $G = (V, E)$
  \pause
  \item Ziel: Finde eine möglichst kurze „Rundreise“
  \pause
  \item Bedingung 1: Besuche jeden Knoten aus $G$ nur einmal
  \pause
  \item Bedingung 2: Startknoten $=$ Endknoten
  \pause
\end{itemize}

  \begin{figure}
    \centering
    \begin{tikzpicture}[%
      >=stealth,
      node distance=2cm,
      on grid,
      auto
    ]
    \node[state] (A)              {A};
    \node[state] (B) [right of=A] {B};
    \node[state] (C) [right of=B] {C};
    \node[state] (D) [right of=C] {D};
    \node[state] (E) [right of=D] {E};
    \path[->] (A) edge node {9} (B);
    \path[->] (B) edge node {2} (C);
    \path[->] (C) edge node {4} (D);
    \path[->] (D) edge node {5} (E);
    \path[->] (E) edge [bend left=25] node  {22} (A);
    \end{tikzpicture}
  \end{figure}
}

\frame{\frametitle{Abgrenzung TSP-Varianten}
\begin{itemize}
  \item Symmetrisches TSP
  \pause
  \item Asymmetrisches TSP
  \pause
  \item metrisches TSP
\end{itemize}
}

\frame{\frametitle{Symmetrisches TSP}
  \begin{itemize}
    \item Kantengewichte zwischen $A$ und $B$ sind gleich.
    \pause
  \end{itemize}
  \begin{figure}
    \centering
  \begin{tikzpicture}[%
    >=stealth,
    node distance=2cm,
    on grid,
    auto
  ]
    \node[state] (A)              {A};
    \node[state] (B) [right of=A] {B};
    \path[->] (A) edge [bend right=-20] node {9} (B);
    \path[->] (B) edge [bend left=20] node  {9} (A);
  \end{tikzpicture}
  \end{figure}
}

\frame{\frametitle{Asymmetrisches TSP}
  \begin{itemize}
    \item Kantengewichte, um von $A$ nach $B$ zu kommen sind unterschiedlich
    \pause
  \end{itemize}
  \begin{figure}
    \centering
    \begin{tikzpicture}[%
      >=stealth,
      node distance=2cm,
      on grid,
      auto
    ]
    \node[state] (A)              {A};
    \node[state] (B) [right of=A] {B};
    \path[->] (A) edge [bend right=-20] node {9} (B);
    \path[->] (B) edge [bend left=20] node  {5} (A);
    \end{tikzpicture}
  \end{figure}

  \begin{itemize}
    \item Wir suchen Lösungen für dieses Problem!
  \end{itemize}
}

\frame{\frametitle{metrisches TSP}
  \begin{itemize}
    \item\onslide<2-> { Dreiecksungleichung $c \leq a + b$ gilt für alle
      Kantenlängen}
    \item \onslide<3-> {Der direkte Weg von Knoten $A$ nach $B$ ist immer der
      kürzeste.}
    \item \onslide<4-> {Es lohnen sich also keine Umwege}
      \pause
  \end{itemize}
  \onslide<1-> {
  \begin{figure}
    \centering
    \def\svgwidth{220pt}
    \input{triangle.pdf_tex}
  \end{figure}
  }
}

\frame{\frametitle{Komplexität}
  \begin{itemize}
    \item kombinatorisches Problem
    \pause
    \item Anzahl Touren: $(n-1)! = (n-1) \cdot (n-2) \cdot ...
      \cdot 2 \cdot 1$
    \pause
      \item
      $\mathcal{O}{(n!)}$ bzw. $\mathcal{O}(2^n)$
      \pause
    \item Nicht in polynomieller Zeit lösbar, falls
      \textbf{P} $\neq$ \textbf{NP}
      \pause
    \item NP-vollständig
  \end{itemize}
}

\section{Evolutionäre Algorithmen}
\label{sec:evolutionäre}
\frame{\frametitle{Grundlegendes}
  \begin{itemize}
    \item Wie kann man trotzdem „gute“ Lösungen finden?
    \pause
    \item \textbf{Evolutionäre Algorithmen}
    \pause
    \item Heuristik
    \pause
    \item Für jegliche Art von Optimierungsproblemen geeignet
    \pause
    \item Beim TSP: Finde eine möglichst \textbf{optimale} (kurze) Rundreise
\end{itemize}
}

\frame{\frametitle{Optimierungsproblem}
  \begin{itemize}
    \item 3-Tupel $(\Omega, f, \succeq)$
    \pause
    \item $\Omega$, Suchraum, z.B. alle möglichen Rundreisen
    \pause
    \item $f : \Omega \rightarrow \mathbb{R}$, die Fitnessfunktion
        (Definition für TSP später)
    \pause
    \item $\succeq \in \{<, >\}$, Vergleichsrelation, Maximierung oder Minimierung
  \end{itemize}
}

% Definition des Bildes des evolutionären Ablaufs.
\newsavebox{\evolpicture}
\begin{lrbox}{\evolpicture}
  \begin{tikzpicture}[%
    >=stealth,
    node distance=1.5cm,
    on grid,
    auto
  ]
    \node[elliptic state] (A)    {Initialisierung};
    \node[elliptic state] (B) [below=1.3cm] {Abbruchbedigung};
    \node[elliptic state] (C) [below right=1.3cm and 3cm of B] {Paarungsselektion};
    \node[elliptic state] (D) [below=of C] {Rekombination};
    \node[elliptic state] (E) [below left=1.3cm and 3cm of D] {Mutation};
    \node[elliptic state] (F) [above left=1.3cm and 3cm of E] {Bewertung};
    \node[elliptic state] (G) [above=of F] {Umweltselektion};
    \path[->] (A) edge node {} (B);
    \path[->] (B) edge node {} (C);
    \path[->] (C) edge node {} (D);
    \path[->] (D) edge node {} (E);
    \path[->] (E) edge node {} (F);
    \path[->] (F) edge node {} (G);
    \path[->] (G) edge node {} (B);

  \end{tikzpicture}
\end{lrbox}

% Command für das Erstellen des Bildes des evolutionären Ablaufs an der
% Seite
% Mit \Highlight können hier Nodes hervorgehoben werden
\newcommand\HighlightedNode{none}
\newcommand\Highlight[1]{\renewcommand\HighlightedNode{#1}}
\long\def\ifnodedefined#1#2#3{%
    \@ifundefined{pgf@sh@ns@#1}{#3}{#2}%
}
\newcommand \evolpictureside {
  \begin{tikzpicture}[%
    >=stealth,
    node distance=1.5cm,
    on grid,
    auto
  ]
    \node[rect state] (A)    {Initialisierung};
    \node[rect state] (B) [below of = A] {Abbruchbedigung};
    \node[rect state] (C) [below of = B] {Paarungsselektion};
    \node[rect state] (D) [below of = C] {Rekombination};
    \node[rect state] (E) [below of = D] {Mutation};
    \node[rect state] (F) [below of = E] {Bewertung};
    \node[rect state] (G) [below of = F] {Umweltselektion};
    \path[->] (A) edge node {} (B);
    \path[->] (B) edge node {} (C);
    \path[->] (C) edge node {} (D);
    \path[->] (D) edge node {} (E);
    \path[->] (E) edge node {} (F);
    \path[->] (F) edge node {} (G);
    %\path[->] (G) edge node {} (B);

    \ifnodedefined{\HighlightedNode}{
      \draw [ultra thick,hsrmRed] (\HighlightedNode.center) circle [x radius=5em,y radius=1.5em];}{}
  \end{tikzpicture}
}

\frame{\frametitle{Evolutionärer Ablauf}
  \begin{figure}
    \centering
    \usebox{\evolpicture}
  \end{figure}
}

\frame{\frametitle{Initialisierung}
  \begin{columns}[T]
    \begin{column}{.75\textwidth}
% TEXT
        \begin{itemize}
          \item \onslide<2-> {Erzeugen der ersten Individuen}
          \item \onslide<3-> {Individuen $\equiv$ Rundreisen}
          \item \onslide<4-> {Ziel: Rundreisen versuchen zu verbessern}
        \end{itemize}
    \end{column}
    \begin{column}{.25\textwidth}
% IMAGE
      \onslide<1-> {
        \resizebox{85pt}{180pt}{%
          \Highlight{A}
          \evolpictureside
        }
      }
    \end{column}
  \end{columns}
}

\frame{\frametitle{Abbruchbedingung}
  \begin{columns}[T]
    \begin{column}{.75\textwidth}
% TEXT
      \begin{itemize}
        \item \onslide<2-> {Abbruch nach $n$ Generationen}
        \item \onslide<3-> {Generation: Eltern $+$ Kinder nach einem
          Durchlauf}
      \end{itemize}
    \end{column}
    \begin{column}{.25\textwidth}
% IMAGE
      \onslide<1-> {
      \resizebox{85pt}{180pt}{%
          \Highlight{B}
          \evolpictureside
        }
      }
    \end{column}
  \end{columns}
}

\frame{\frametitle{Paarungsselektion}
  \begin{columns}[T]
    \begin{column}{.75\textwidth}
% TEXT
      \begin{itemize}
        \item \onslide<2-> {Vorhanden: Pool von Rundreisen}
        \item \onslide<3-> {Frage: Welche Rundreise mit welcher paaren?}
        \item \onslide<4-> {Möglichkeiten:
          \begin{itemize}
            \item Random
            \item Turnierselektion
          \end{itemize} }
      \end{itemize}

    \end{column}
    \begin{column}{.25\textwidth}
% IMAGE
      \onslide<1-> {
      \resizebox{85pt}{180pt}{%
          \Highlight{C}
          \evolpictureside
        }
      }
    \end{column}
  \end{columns}
}

\frame{\frametitle{Rekombination}
  \begin{columns}[T]
    \begin{column}{.75\textwidth}
% TEXT
      \begin{itemize}
        \item \onslide<2-> {Wir wollen zwei Rundreisen $G_1$ und $G_2$
          paaren.}
        \item \onslide<3-> {Ergebnis ist ein Kind oder Offspring $G_o$}
        \item \onslide<4-> {Frage: Wie machen wir das?}
        \item \onslide<5-> {verschiedene Strategien (Crossover)}
          \begin{itemize}
            \item \onslide<6-> {Kantenrekombination}
            \item \onslide<7-> {GAPX}
          \end{itemize}
      \end{itemize}
    \end{column}
    \begin{column}{.25\textwidth}
% IMAGE
      \onslide<1-> {
      \resizebox{85pt}{180pt}{%
          \Highlight{D}
          \evolpictureside
        }
      }
    \end{column}
  \end{columns}
}

\frame{\frametitle{GAPX}
  \begin{columns}[T]
    \begin{column}{.75\textwidth}
% TEXT
      \begin{itemize}
        \item\onslide<2->{ „Generalized Asymmetric Partition Crossover“}
        \item\onslide<3->{Speziell für ATSP}
      \end{itemize}

    \end{column}
    \begin{column}{.25\textwidth}
% IMAGE
      \onslide<1->{
      \resizebox{85pt}{180pt}{%
          \Highlight{D}
          \evolpictureside
        }
      }
    \end{column}
  \end{columns}
}

\frame{\frametitle{GAPX: Partition-Crossover}
  \begin{columns}[T]
    \begin{column}{.75\textwidth}
% TEXT
      \vspace{0.2cm}
      \begin{columns}[T]
        \begin{column}{.5\textwidth}
    \onslide<2-> {
    \resizebox{120pt}{150pt}{%
      \begin{tikzpicture}[%
        >=stealth,
        node distance=2cm,
        on grid,
        auto
      ]
        \node[state] (1)              {1};


        \node[state] (3) [above right of=1]             {3};
        \node[state] (2) [below right of=1]            {2};
        \node[state] (4) [right of=3]             {4};
        \node[state] (5) [right of=2]             {5};

        \node[state] (6) [below right of=5] {6};

        \node[state] (7) [below left of=6] {7};
        \node[state] (10) [left of=7] {10};

        \node[state] (11) [below left of=10] {11};

        \node[state] (9) [below right of=11] {9};
        \node[state] (8) [right of=9] {8};

        % solid: parent 1
        \path[->] (1) edge [blue, bend left=0] node  {} (3);
        \path[->] (3) edge [blue, bend left=0] node  {} (2);
        \path[->] (2) edge [blue, bend left=0] node  {} (5);
        \path[->] (5) edge [blue, bend left=0] node  {} (4);
        \path[->] (4) edge [blue, bend left=0] node  {} (6);
        \path[->] (6) edge [blue, bend left=0] node  {} (8);
        \path[->] (8) edge [blue, bend left=0] node  {} (7);
        \path[->] (7) edge [blue, bend left=0] node  {} (10);
        \path[->] (10) edge [blue, bend left=0] node  {} (9);
        \path[->] (9) edge [blue, bend left=0] node  {} (11);
        \path[->] (11) edge [blue, bend left=0] node  {} (1);

        % dashed: parent 2
        \path[->] (1) edge [bend left=0, dashed, hsrmRed] node {} (2);
        \path[->] ([xshift=0.7ex] 2.north) edge [hsrmRed, dashed] node {}
                 ([xshift=0.7ex] 3.south);
        \path[->] (3) edge [hsrmRed, dashed] node {}
                 (4);
        \path[->] ([xshift=0.7ex] 4.south) edge [hsrmRed, dashed] node {}
                 ([xshift=0.7ex] 5.north);
        \path[->] (5) edge [hsrmRed, dashed] node {}
                 (6);
        \path[->] (6) edge [hsrmRed, dashed] node {}
                 (7);
        \path[->] ([xshift=0.7ex] 7.south) edge [hsrmRed, dashed] node {}
                 ([xshift=0.7ex] 8.north);
        \path[->] (8) edge [hsrmRed, dashed] node {}
                 (9);
        \path[->] ([xshift=0.7ex] 9.north) edge [hsrmRed, dashed] node {}
                 ([xshift=0.7ex] 10.south);
        \path[->] (10) edge [hsrmRed, dashed] node {}
                 (11);
        \path[->] ([xshift=0.7ex] 11.north) edge [hsrmRed, dashed] node {}
                 ([xshift=0.7ex] 1.south);
      \end{tikzpicture}
      }
      }


        \end{column}
        \begin{column}{.5\textwidth}

        \onslide<3-> {
            \resizebox{120pt}{150pt}{%
      \begin{tikzpicture}[%
        >=stealth,
        node distance=2cm,
        on grid,
        auto
      ]
        \node[state] (1)              {1};


        \node[state] (3) [above right of=1]             {3};
        \node[state] (2) [below right of=1]            {2};
        \node[state] (4) [right of=3]             {4};
        \node[state] (5) [right of=2]             {5};

        \node[state] (6) [above right of=5] {6};

        \node[state] (7) [below of=5] {7};
        \node[state] (6') [below right of=7] {6'};
        \node[state] (10) [left of=7] {10};

        \node[state] (11) [below left of=10] {11};

        \node[state] (9) [below right of=11] {9};
        \node[state] (8) [right of=9] {8};

        % solid: parent 1
        \path[->] (1) edge [blue, bend left=0] node  {} (3);
        \path[->] (3) edge [blue, bend left=0] node  {} (2);
        \path[->] (2) edge [blue, bend left=0] node  {} (5);
        \path[->] (5) edge [blue, bend left=0] node  {} (4);
        \path[->] (6') edge [blue, bend left=0] node  {} (8);
        \path[->] (4) edge [blue, bend left=0] node  {} (6);
        \path[->] (8) edge [blue, bend left=0] node  {} (7);
        \path[->] (7) edge [blue, bend left=0] node  {} (10);
        \path[->] (10) edge [blue, bend left=0] node  {} (9);
        \path[->] (9) edge [blue, bend left=0] node  {} (11);
        \path[->] (11) edge [blue, bend left=0] node  {} (1);
        \path[->] (6) edge [blue, bend left=0] node  {} (6');

        % dashed: parent 2
        \path[->] (1) edge [bend left=0, dashed, hsrmRed] node {} (2);
        \path[->] ([xshift=0.7ex] 2.north) edge [hsrmRed, dashed] node {}
                 ([xshift=0.7ex] 3.south);
        \path[->] (3) edge [hsrmRed, dashed] node {}
                 (4);
        \path[->] ([xshift=0.7ex] 4.south) edge [hsrmRed, dashed] node {}
                 ([xshift=0.7ex] 5.north);
        \path[->] (5) edge [hsrmRed, dashed] node {}
                 (6);
        \path[->] (6') edge [hsrmRed, dashed] node {}
                 (7);
        \path[->] ([xshift=0.7ex] 7.south) edge [hsrmRed, dashed] node {}
                 ([xshift=0.7ex] 8.north);
        \path[->] (8) edge [hsrmRed, dashed] node {}
                 (9);
        \path[->] ([xshift=0.7ex] 9.north) edge [hsrmRed, dashed] node {}
                 ([xshift=0.7ex] 10.south);
        \path[->] (10) edge [hsrmRed, dashed] node {}
                 (11);
        \path[->] ([xshift=0.7ex] 11.north) edge [hsrmRed, dashed] node {}
                 ([xshift=0.7ex] 1.south);
        \path[->] ([xshift=0.7ex] 6.south) edge [hsrmRed, dashed] node {}
                 ([xshift=0.7ex] 6'.north);


      \end{tikzpicture}
      }
      }

        \end{column}
      \end{columns}
      \begin{itemize}
        \item\onslide<2->{ Gegeben sind zwei Graphen
          $\begingroup\color{blue}G_1\endgroup$
          und
          $\begingroup\color{hsrmRed}G_2\endgroup$}
        \item\onslide<3->{ Ghost-Knoten $v'$ in
          $\begingroup\color{blue}G_1\endgroup \cup
          \begingroup\color{hsrmRed}G_2\endgroup$ einfügen, für alle Knoten $v$
          mit $deg(v) = 4$}
      \end{itemize}
    \end{column}
    \begin{column}{.25\textwidth}
% IMAGE
      \onslide<1-> {
      \resizebox{85pt}{180pt}{%
          \Highlight{D}
          \evolpictureside
        }
      }
    \end{column}
  \end{columns}
}

\frame{\frametitle{GAPX: Partition-Crossover}
  \begin{columns}[T]
    \begin{column}{.75\textwidth}
% TEXT

      \vspace{0.2cm}

      \begin{columns}[T]
        \begin{column}{.5\textwidth}

      \onslide<1-> {
      \resizebox{120pt}{150pt}{%
      \begin{tikzpicture}[%
        >=stealth,
        node distance=2cm,
        on grid,
        auto
      ]
        \node[state] (1)              {1};


        \node[state] (3) [above right of=1]             {3};
        \node[state] (2) [below right of=1]            {2};
        \node[state] (4) [right of=3]             {4};
        \node[state] (5) [right of=2]             {5};

        \node[state] (6) [above right of=5] {6};

        \node[state] (7) [below of=5] {7};
        \node[state] (6') [below right of=7] {6'};
        \node[state] (10) [left of=7] {10};

        \node[state] (11) [below left of=10] {11};

        \node[state] (9) [below right of=11] {9};
        \node[state] (8) [right of=9] {8};

        % solid: parent 1
        \path[->] (1) edge [blue, bend left=0] node  {} (3);
        \path[->] (3) edge [blue, bend left=0] node  {} (2);
        \path[->] (2) edge [blue, bend left=0] node  {} (5);
        \path[->] (5) edge [blue, bend left=0] node  {} (4);
        \path[->] (6') edge [blue, bend left=0] node  {} (8);
        \path[->] (4) edge [blue, bend left=0] node  {} (6);
        \path[->] (8) edge [blue, bend left=0] node  {} (7);
        \path[->] (7) edge [blue, bend left=0] node  {} (10);
        \path[->] (10) edge [blue, bend left=0] node  {} (9);
        \path[->] (9) edge [blue, bend left=0] node  {} (11);
        %\path[->] (11) edge [blue, bend left=0] node  {} (1);
        %\path[->] (6) edge [blue, bend left=0] node  {} (6');

        % dashed: parent 2
        \path[->] (1) edge [bend left=0, dashed, hsrmRed] node {} (2);
        \path[->] ([xshift=0.7ex] 2.north) edge [hsrmRed, dashed] node {}
                 ([xshift=0.7ex] 3.south);
        \path[->] (3) edge [hsrmRed, dashed] node {}
                 (4);
        \path[->] ([xshift=0.7ex] 4.south) edge [hsrmRed, dashed] node {}
                 ([xshift=0.7ex] 5.north);
        \path[->] (5) edge [hsrmRed, dashed] node {}
                 (6);
        \path[->] (6') edge [hsrmRed, dashed] node {}
                 (7);
        \path[->] ([xshift=0.7ex] 7.south) edge [hsrmRed, dashed] node {}
                 ([xshift=0.7ex] 8.north);
        \path[->] (8) edge [hsrmRed, dashed] node {}
                 (9);
        \path[->] ([xshift=0.7ex] 9.north) edge [hsrmRed, dashed] node {}
                 ([xshift=0.7ex] 10.south);
        \path[->] (10) edge [hsrmRed, dashed] node {}
                 (11);
        %\path[->] ([xshift=0.7ex] 11.north) edge [hsrmRed, dashed] node {}
        %         ([xshift=0.7ex] 1.south);
        %\path[->] ([xshift=0.7ex] 6.south) edge [hsrmRed, dashed] node {}
        %         ([xshift=0.7ex] 6'.north);


        \end{tikzpicture}
        }
        }

        \end{column}

        \begin{column}{.5\textwidth}

      \onslide<2-> {
      \resizebox{120pt}{150pt}{%
      \begin{tikzpicture}[%
        >=stealth,
        node distance=2cm,
        on grid,
        auto
      ]
        \node[state] (1)              {1};


        \node[state] (3) [above right of=1]             {3};
        \node[state] (2) [below right of=1]            {2};
        \node[state] (4) [right of=3]             {4};
        \node[state] (5) [right of=2]             {5};

        \node[state] (6) [above right of=5] {6};

        \node[state] (7) [below of=5] {7};
        \node[state] (6') [below right of=7] {6'};
        \node[state] (10) [left of=7] {10};

        \node[state] (11) [below left of=10] {11};

        \node[state] (9) [below right of=11] {9};
        \node[state] (8) [right of=9] {8};

        % solid: parent 1
        \path[->] (1) edge [blue, bend left=0] node  {} (3);
        \path[->] (3) edge [blue, bend left=0] node  {} (2);
        \path[->] (2) edge [blue, bend left=0] node  {} (5);
        \path[->] (5) edge [blue, bend left=0] node  {} (4);
        %\path[->] (6') edge [blue, bend left=0] node  {} (8);
        \path[->] (4) edge [blue, bend left=0] node  {} (6);
        %\path[->] (8) edge [blue, bend left=0] node  {} (7);
        %\path[->] (7) edge [blue, bend left=0] node  {} (10);
        %\path[->] (10) edge [blue, bend left=0] node  {} (9);
        %\path[->] (9) edge [blue, bend left=0] node  {} (11);
        \path[->] (11) edge [blue, bend left=0] node  {} (1);
        \path[->] (6) edge [blue, bend left=0] node  {} (6');

        % dashed: parent 2
        %\path[->] (1) edge [bend left=0, dashed, hsrmRed] node {} (2);
        %\path[->] ([xshift=0.7ex] 2.north) edge [hsrmRed, dashed] node {}
        %         ([xshift=0.7ex] 3.south);
        %\path[->] (3) edge [hsrmRed, dashed] node {}
        %         (4);
        %\path[->] ([xshift=0.7ex] 4.south) edge [hsrmRed, dashed] node {}
        %         ([xshift=0.7ex] 5.north);
        %\path[->] (5) edge [hsrmRed, dashed] node {}
        %         (6);
        \path[->] (6') edge [hsrmRed, dashed] node {}
                 (7);
        \path[->] ([xshift=0.7ex] 7.south) edge [hsrmRed, dashed] node {}
                 ([xshift=0.7ex] 8.north);
        \path[->] (8) edge [hsrmRed, dashed] node {}
                 (9);
        \path[->] ([xshift=0.7ex] 9.north) edge [hsrmRed, dashed] node {}
                 ([xshift=0.7ex] 10.south);
        \path[->] (10) edge [hsrmRed, dashed] node {}
                 (11);
        \path[->] ([xshift=0.7ex] 11.north) edge [hsrmRed, dashed] node {}
                 ([xshift=0.7ex] 1.south);
        \path[->] ([xshift=0.7ex] 6.south) edge [hsrmRed, dashed] node {}
                 ([xshift=0.7ex] 6'.north);


      \end{tikzpicture}
      }
      }
    \end{column}
  \end{columns}


      \begin{itemize}
        \item \onslide<1-> {Löschen von Kanten, bei denen
          $\begingroup\color{blue}G_1\endgroup$
          und $\begingroup\color{hsrmRed}G_2\endgroup$ in die selbe
          Richtung zeigen }
        \item \onslide<2-> {Erzeugung des Offsprings $G_o$ aus $\begingroup\color{blue}G_1\endgroup \cup
          \begingroup\color{hsrmRed}G_2\endgroup$}
      \end{itemize}
    \end{column}
    \begin{column}{.25\textwidth}
% IMAGE

      \onslide<1-> {
      \resizebox{85pt}{180pt}{%
          \Highlight{D}
          \evolpictureside
        }
      }
    \end{column}
  \end{columns}
}

\frame{\frametitle{Mutation}
  \begin{columns}[T]
    \begin{column}{.75\textwidth}
% TEXT
      \begin{itemize}
        \item \onslide<2-> {Teile einer Rundreise stärker verändern}
        \item \onslide<3-> {Man möchte nicht im lokalen Optimum gefangen
          sein}
        \item \onslide<4-> {Möglichkeiten}
          \begin{itemize}
            \item \onslide<5-> {vertauschende Mutation
              $(1,\mathunderline{hsrmRed}{2},3,4,5,\mathunderline{hsrmRed}{6},7,8)
              \rightarrow
              (1,\mathunderline{hsrmRed}{6},3,4,5,\mathunderline{hsrmRed}{2},7,8)$}
            \item \onslide<6-> {invertierende Mutation
              $(1,\mathunderline{hsrmRed}{2},3,4,5,\mathunderline{hsrmRed}{6},7,8)
              \rightarrow (1,\mathunderline{hsrmRed}{6,5,4,3,2},7,8)$}
          \end{itemize}
      \end{itemize}
    \end{column}
    \begin{column}{.25\textwidth}
% IMAGE
      \onslide<1-> {
      \resizebox{85pt}{180pt}{%
          \Highlight{E}
          \evolpictureside
        }
      }
    \end{column}
  \end{columns}
}

\frame{\frametitle{Bewertung mittels Fitness-Funktion}
  \begin{columns}[T]
    \begin{column}{.75\textwidth}
% TEXT
      \begin{itemize}
        \item \onslide<2-> {Gegeben sei eine Rundreise $G=(V, E\,)$}
        \item \onslide<3-> {Kantengewicht: $d(e)$, mit $e \in E$}
      \end{itemize}

      \onslide<4-> {
      \begin{align*}
        f(G) = \sum_{e \in E} d(e)
      \end{align*}
      }

      \begin{itemize}
        \item \onslide<5-> {Wir summieren alle Kantengewichte}
        \item \onslide<6-> {Beste Lösung:}
      \end{itemize}
      \onslide<7-> {
      \begin{align*}
        f(G\,) = \min\{\, f(x) \mid x \in \Omega\,\}, G \in \Omega
      \end{align*}
      }

    \end{column}
    \begin{column}{.25\textwidth}
% IMAGE
      \onslide<1-> {
      \resizebox{85pt}{180pt}{%
          \Highlight{F}
          \evolpictureside
        }
      }
    \end{column}
  \end{columns}
}

\frame{\frametitle{Umweltselektion}
  \begin{columns}[T]
    \begin{column}{.75\textwidth}
% TEXT
      \begin{itemize}
        \item \onslide<2-> {Frage: Wer kommt in die nächste Generation}
        \item \onslide<3-> {Möglichkeiten:}
        \begin{itemize}
          \item \onslide<4-> {Nur erzeugte Kinder}
          \item \onslide<5-> {Kinder $+$ Eltern}
        \end{itemize}
      \end{itemize}
    \end{column}
    \begin{column}{.25\textwidth}
% IMAGE
      \onslide<1-> {
      \resizebox{85pt}{180pt}{%
          \Highlight{G}
          \evolpictureside
        }
      }
    \end{column}
  \end{columns}
}

\section{Erlang}
\label{sec:erlang}

\begin{frame}
  \frametitle{Was ist Erlang?}

  Erlang ist, eine funktionale Programmiersprache für massiv
  skalierbare und distributierte Systeme.\footnote{\url{http://www.erlang.org/}}

  Wir haben uns für Erlang entschieden, da Parallelisierung mithilfe
  von „message passing“ ein Grundprinzip der Sprache ist, was auch auf
  mehrere Systeme ausgeweitet werden kann.
\end{frame}

\begin{frame}
  \frametitle{Andere Charakteristika}

  Weitere Eigenschaften:
  \begin{itemize}
  \item soft real time
  \item hot code loading
  \item hohe Fehlertoleranz
  \item garbage collected
  \item läuft auf einer virutellen Maschine (BEAM)
  \end{itemize}

  Ursprünglich wurde die Sprache für langläufige
  Telekommunikationssysteme entwickelt, wo diese Punkte eine deutlich
  größere Rolle spielen.
\end{frame}

\begin{frame}
  \frametitle{BEAM}

  Ist ein Prozess in dem Betriebssystem.  Erstellt pro Prozessorkern
  einen Thread.

  Benutzt eine eigene run-queue.  Wurde ähnlich einem Betriebssystem
  konzipiert; es gibt sogar „Erlang on Xen“ – Erlang läuft als eigenes
  Betriebssystem auf bare metal.
\end{frame}

\begin{frame}
  \frametitle{Was ist funktionale Programmierung?}

  FP ist an mathematscher Notation angelehnt.

  Im idealen Sinn haben Funktionen keine Seiteneffekte.

  Daten können nicht verändert werden. % immutability
\end{frame}

\begin{frame}
  \frametitle{Seiteneffekte}

  \begin{block}{Referentielle Transparenz}
    Ausdruck kann durch sein Ergebnis ersetzt werden ohne die
    Korrektheit des Programms zu verändern.
  \end{block}
  Das impliziert Determinismus und ist unter anderem wichtig für
  Compiler-Optimierungen.  Außerdem ermöglicht es Memoisierung.
  % Cachen von Funktionsergebnissen

  \bigskip Im Allgemeinen werden Funktionen nur mit „call-by-value“
  aufgerufen.  Allerdings wird das Konzept teilweise gebrochen, um
  z. B. IO zu realisieren.
\end{frame}

\begin{frame}
  \frametitle{Immutable data}

  Variablen können nach dem Binden nicht mehr verändert werden. Zum
  Beipiel ist \lstinline{i = i + 1} nicht möglich in Erlang.

  \pause
  \bigskip Sieht $n=n+1$ natürlich aus?
  % Evtl. ist ein Beispiel eingänglicher.

\end{frame}

\begin{frame}[fragile]
  \frametitle{Elemente der Sprache}

  Variablen fangen mit einem Großbuchstaben an.  Kleingeschriebene
  Wörter sind {\Medium Atome}.

  Ein Tupel $(km, 20)$ wird in Erlang \verb|{km, 20}| geschrieben.

  Listen sehen so aus: \quad\verb|List = [this, is, a, list].|
\end{frame}

\begin{frame}[fragile]
  \frametitle{}
  $$
  f(n) =
  \begin{cases}
    1             &\textrm{wenn } n = 0 \\
    n \cdot f(n - 1) & \textrm{sonst}
  \end{cases}
  $$
  \vfill

  \begin{lstlisting}
    f(0) -> 1;
    f(N) -> N * f(N - 1).
  \end{lstlisting}
\end{frame}

\begin{frame}[fragile]
  \frametitle{Listen und Mengen}
  $$ \{\,2 \cdot x \mid x \in X\,\}$$

  % \vfill

  % \begin{lstlisting}
  %   [2 * X || X <- Xs].
  % \end{lstlisting}
\end{frame}

\section{Literatur}
\label{sec:literatur}


\frame { \frametitle{Literatur I}
  \bibliographystyle{plain}
  \nocite{*}
  \tiny\bibliography{literatur}
}

\end{document}
