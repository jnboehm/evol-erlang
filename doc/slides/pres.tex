\documentclass[compress]{beamer}

% basically only used for the \to in \item
\usepackage[math]{kurier}
\usepackage{listings}
\usepackage[ngerman]{babel}
\usepackage{qtree}
\usepackage{fontspec}
\usepackage{xcolor}
\usepackage{tikz}
\usepackage{graphicx}
\usepackage{graphviz}
\usepackage{auto-pst-pdf}
\usepackage{tikz}
\usetikzlibrary{shapes,shapes.geometric,arrows,fit,calc,positioning,automata}

% verschiedene Shapes für die Zustände
\tikzset{elliptic state/.style={draw,ellipse}}
\tikzset{rect state/.style={draw,rectangle}}


\usepackage[noflama]{hsrm}


\lstset{%
  basicstyle=\small\ttfamily, %
  commentstyle=\color{gray}, %
  stringstyle=\color{Green}, %
  keywordstyle=\bfseries\color{hsrmRed}, %
  frame=tb, % line at Top and one at Bottom
  showstringspaces=false, %
  language=erlang, %
  moredelim=[is][\textcolor{gray}]{\%\%}{\%\%}, %colorize opt params
  moredelim=[is][\textcolor{hsrmSec2Dark}]{==}{==}, %To overcome the weird
  % behavior of Strings in
  % beamer
}

\newcommand{\hsrm}{Hochschule {\Medium RheinMain}}

\AtBeginDocument{
  \title{Traveling Salesman Problem}
  \subtitle{Evolutionäre Algorithmen in Erlang}
  \author{Niklas Böhm \& Jens Nazarenus}
  \institute{%
    \begin{tabular}{r l } %{\Medium}
      Fachbereich & {\Medium DCSM} \\
      Studiengang & {\Medium Angewandte Informatik}
    \end{tabular}%
  }
  \date{\today}
}

\begin{document}
\maketitle

\begin{frame}{Gliederung}
  \tableofcontents[hideallsubsections]
\end{frame}

\section{Traveling Salesman Problem}
\label{sec:tsp}

\frame{\frametitle{Problemvorstellung} 
\begin{itemize}
  \item Gegeben sei ein Graph $G$
  \pause
  \item Ziel: Finde eine möglichst kurze „Rundreise“
  \pause
  \item Bedingung 1: Besuche jeden Knoten aus $G$ nur einmal  
  \pause
  \item Bedingung 2: Startknoten $=$ Endknoten
  \pause
\end{itemize}

  \begin{figure}
    \centering
    \begin{tikzpicture}[%
      >=stealth,
      node distance=2cm,
      on grid,
      auto
    ]
    \node[state] (A)              {A};
    \node[state] (B) [right of=A] {B};
    \node[state] (C) [right of=B] {C};
    \node[state] (D) [right of=C] {D};
    \node[state] (E) [right of=D] {E};
    \path[->] (A) edge node {9} (B);
    \path[->] (B) edge node {2} (C);
    \path[->] (C) edge node {4} (D);
    \path[->] (D) edge node {5} (E);
    \path[->] (E) edge [bend left=25] node  {22} (A);
    \end{tikzpicture}
  \end{figure}
}

\frame{\frametitle{Abgrenzung TSP-Varianten}
\begin{itemize}
  \item Symmetrisches TSP 
  \pause
  \item Asmymmetrisches TSP
  \pause
  \item metrisches TSP
\end{itemize}
}

\frame{\frametitle{Symmetrisches TSP}
  \begin{itemize}
    \item Kantengewichte zwischen $A$ und $B$ sind gleich. 
    \pause
  \end{itemize}
  \begin{figure}
    \centering
  \begin{tikzpicture}[%
    >=stealth,
    node distance=2cm,
    on grid,
    auto
  ]
    \node[state] (A)              {A};
    \node[state] (B) [right of=A] {B};
    \path[->] (A) edge [bend right=-20] node {9} (B);
    \path[->] (B) edge [bend left=20] node  {9} (A);
  \end{tikzpicture}
  \end{figure}
}

\frame{\frametitle{Asymmetrisches TSP}
  \begin{itemize}
    \item Kantengewichte, um von $A$ nach $B$ zu kommen sind unterschiedlich
    \pause
  \end{itemize}
  \begin{figure}
    \centering
    \begin{tikzpicture}[%
      >=stealth,
      node distance=2cm,
      on grid,
      auto
    ]
    \node[state] (A)              {A};
    \node[state] (B) [right of=A] {B};
    \path[->] (A) edge [bend right=-20] node {9} (B);
    \path[->] (B) edge [bend left=20] node  {5} (A);
    \end{tikzpicture}
  \end{figure}
  
  \begin{itemize}
    \item Wir suchen Lösungen für dieses Problem!
  \end{itemize}
}


\frame{\frametitle{metrisches TSP}
  \begin{itemize}
      \item Dreiecksungleichung $c \leq a + b$ gilt für alle
        Kantenlängen
      \pause
      \item Der direkte Weg von Knoten $A$ nach $B$ ist immer der
        kürzeste.
      \pause
      \item Es lohnen sich also keine Umwege
      \pause
  \end{itemize}
  \begin{figure}
    \centering
    \def\svgwidth{220pt}
    \input{triangle.pdf_tex}
  \end{figure}
}

\frame{\frametitle{Komplexität}
  \begin{itemize}
    \item kombinatorisches Problem
    \pause
    \item Anzahl Touren: $(n-1)! = (n-1) \cdot (n-2) \cdot ...
      \cdot 2 \cdot 1$
    \pause
    \item Nicht in polynomieller Zeit lösbar
    \pause
    \item stark NP-vollständig
    \pause
    \item Komplexität direkte Lösung: $\mathcal{O}(n!)$
  \end{itemize}
}

\section{Evolutionäre Algorithmen}
\label{sec:evolutionäre}
\frame{\frametitle{Grundlegendes}
  \begin{itemize}
    \item Evolutionärer Ansatz ist eine Heuristik
    \item Für jegliche Art von Optimierungsproblemen geeignet
    \item Beim TSP: Finde eine möglichst \textbf{optimale} (kurze) Rundreise
  \end{itemize}
}

\frame{\frametitle{Optimierungsproblem}
  \begin{itemize}
    \item 3-Tupel $(\Omega, f, \{<, >\})$
      \item $\Omega$, Suchraum, z.B. alle möglichen Rundreisen
      \item $f : \Omega \rightarrow \mathbb{R}$, die Fitnessfunktion
        (Definition für TSP später)
      \item $<, >$, Vergleichsrelation, Maximierung oder Minimierung
  \end{itemize}
}

% Definition des Bildes des evolutionären Ablaufs. 
\newsavebox{\evolpicture}
\begin{lrbox}{\evolpicture}
  \begin{tikzpicture}[%
    >=stealth,
    node distance=1.5cm,
    on grid,
    auto
  ]
    \node[elliptic state] (A)    {Initialisierung};
    \node[elliptic state] (B) [below=1.3cm] {Abbruchbedigung};
    \node[elliptic state] (C) [below right=1.3cm and 3cm of B] {Paarungsselektion};
    \node[elliptic state] (D) [below=of C] {Rekombination};
    \node[elliptic state] (E) [below left=1.3cm and 3cm of D] {Mutation};
    \node[elliptic state] (F) [above left=1.3cm and 3cm of E] {Bewertung};
    \node[elliptic state] (G) [above=of F] {Umweltselektion};
    \path[->] (A) edge node {} (B);
    \path[->] (B) edge node {} (C);
    \path[->] (C) edge node {} (D);
    \path[->] (D) edge node {} (E);
    \path[->] (E) edge node {} (F);
    \path[->] (F) edge node {} (G);
    \path[->] (G) edge node {} (B);

  \end{tikzpicture}
\end{lrbox}

% Command für das Erstellen des Bildes des evolutionären Ablaufs an der
% Seite
% Mit \Highlight können hier Nodes hervorgehoben werden
\newcommand\HighlightedNode{none}
\newcommand\Highlight[1]{\renewcommand\HighlightedNode{#1}}
\long\def\ifnodedefined#1#2#3{%
    \@ifundefined{pgf@sh@ns@#1}{#3}{#2}%
}
\newcommand \evolpictureside {
  \begin{tikzpicture}[%
    >=stealth,
    node distance=1.5cm,
    on grid,
    auto
  ]
    \node[rect state] (A)    {Initialisierung};
    \node[rect state] (B) [below of = A] {Abbruchbedigung};
    \node[rect state] (C) [below of = B] {Paarungsselektion};
    \node[rect state] (D) [below of = C] {Rekombination};
    \node[rect state] (E) [below of = D] {Mutation};
    \node[rect state] (F) [below of = E] {Bewertung};
    \node[rect state] (G) [below of = F] {Umweltselektion};
    \path[->] (A) edge node {} (B);
    \path[->] (B) edge node {} (C);
    \path[->] (C) edge node {} (D);
    \path[->] (D) edge node {} (E);
    \path[->] (E) edge node {} (F);
    \path[->] (F) edge node {} (G);
    %\path[->] (G) edge node {} (B);

    \ifnodedefined{\HighlightedNode}{
      \draw [ultra thick,hsrmRed] (\HighlightedNode.center) circle [x radius=5em,y radius=1.5em];}{}
  \end{tikzpicture}
}

\frame{\frametitle{Evolutionärer Ablauf}
  \begin{figure}
    \centering
    \usebox{\evolpicture}
  \end{figure}
}

\frame{\frametitle{Initialisierung}
  \begin{columns}[T]
    \begin{column}{.80\textwidth}
% TEXT
        \begin{itemize}
          \item Erzeugen der ersten Individuen
          \item Individuen $\equiv$ Rundreisen
          \item Ziel: Rundreisen versuchen zu verbessern
        \end{itemize}
    \end{column}
    \begin{column}{.20\textwidth}
% IMAGE
        \resizebox{70pt}{200pt}{%
          \Highlight{A}
          \evolpictureside
        }
    \end{column}
  \end{columns}
}

\frame{\frametitle{Abbruchbedingung}
  \begin{columns}[T]
    \begin{column}{.80\textwidth}
% TEXT
      \begin{itemize}
        \item Abbruch nach $n$ Generationen \\
        \item Generation: Eltern $+$ Kinder nach einem Durchlauf
      \end{itemize}
    \end{column}
    \begin{column}{.20\textwidth}
% IMAGE
        \resizebox{70pt}{200pt}{%
          \Highlight{B}
          \evolpictureside
        }
    \end{column}
  \end{columns}
}

\frame{\frametitle{Paarungsselektion}
  \begin{columns}[T]
    \begin{column}{.80\textwidth}
% TEXT
      \begin{itemize}
        \item Vorhanden: Pool von Rundreisen
        \item Frage: Welche Rundreise mit welcher paaren?
        \item Möglichkeiten:
          \begin{itemize}
            \item Random
            \item „Gleich gute”
          \end{itemize}
      \end{itemize}
    \end{column}
    \begin{column}{.20\textwidth}
% IMAGE
        \resizebox{70pt}{200pt}{%
          \Highlight{C}
          \evolpictureside
        }
    \end{column}
  \end{columns}
}

\frame{\frametitle{Rekombination}
  \begin{columns}[T]
    \begin{column}{.80\textwidth}
% TEXT
      \begin{itemize}
        \item Wir wollen zwei Rundreisen $u$ und $v$ paaren.
        \item Frage: Wie machen wir das?
        \item verschiedene Strategien (sog. Crossover)
          \begin{itemize}
            \item Kantenrekombination
            \item GAPX
          \end{itemize}
      \end{itemize}
    \end{column}
    \begin{column}{.20\textwidth}
% IMAGE
        \resizebox{70pt}{200pt}{%
          \Highlight{D}
          \evolpictureside
        }
    \end{column}
  \end{columns}
}

\frame{\frametitle{Kantenrekombination}

}

\frame{\frametitle{GAPX}
  \begin{columns}[T]
    \begin{column}{.80\textwidth}
% TEXT
      \begin{itemize}
        \item „Generalized Asymetric Partition Crossover“
        \item Speziell für ATSP
      \end{itemize}
    \end{column}
    \begin{column}{.20\textwidth}
% IMAGE
        \resizebox{70pt}{200pt}{%
          \Highlight{D}
          \evolpictureside
        }
    \end{column}
  \end{columns}
}

\frame{\frametitle{Mutation}
  \begin{columns}[T]
    \begin{column}{.80\textwidth}
% TEXT
      \begin{itemize}
        \item 
      \end{itemize}
    \end{column}
    \begin{column}{.20\textwidth}
% IMAGE
        \resizebox{70pt}{200pt}{%
          \Highlight{E}
          \evolpictureside
        }
    \end{column}
  \end{columns}
}

\frame{\frametitle{Bewertung mittels Fitness-Funktion}
  \begin{columns}[T]
    \begin{column}{.80\textwidth}
% TEXT
      \begin{itemize}
        \item 
      \end{itemize}
    \end{column}
    \begin{column}{.20\textwidth}
% IMAGE
        \resizebox{70pt}{200pt}{%
          \Highlight{F}
          \evolpictureside
        }
    \end{column}
  \end{columns}
}

\frame{\frametitle{Selektion}
  \begin{columns}[T]
    \begin{column}{.80\textwidth}
% TEXT
      \begin{itemize}
        \item 
      \end{itemize}
    \end{column}
    \begin{column}{.20\textwidth}
% IMAGE
        \resizebox{70pt}{200pt}{%
          \Highlight{G}
          \evolpictureside
        }
    \end{column}
  \end{columns}
}

\frame{\frametitle{Coming up}

}

\section{Erlang}
\label{sec:erlang}

\begin{frame}[fragile]
  \frametitle{}
  $$
  f(n) =
  \begin{cases}
    1            \hfill &\textrm{wenn } n = 0 \\
    n \cdot f(n - 1) & \textrm{sonst}
  \end{cases}
  $$
  \vfill

  \begin{lstlisting}
    f(0) -> 1;
    f(N) -> N * f(N - 1).
  \end{lstlisting}
\end{frame}
\section{Parallelisierung}
\label{sec:parallelisierung}

\section{Auswertung der (bisherigen) Ergebnisse}
\label{sec:auswertung}

\end{document}
